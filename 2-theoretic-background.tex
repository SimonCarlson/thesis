 \documentclass[0-thesis.tex]{subfiles}

\begin{document}

\section{Network Stack}
\subsection{CoAP}

\subsection{DTLS}

\section{SUIT}
The IETF SUIT (Software Updates for Internet of Things) working group aims to define a firmware 
update solution that is interoperable and non-proprietary \parencite{suit}. The solution shall 
be usable on Class 1 devices as defined in RFC 7228. These devices feature ~10 KiB of RAM and 
~100 KiB code size, which makes it suitable for this thesis \parencite{rfc7228}. The solution 
may be applied to more powerful devices. The working group does not however try to define 
new transport or discovery mechanism, making their proposal angostic of any particular 
technology. As of writing, they have published two documents concerning the architecture of 
such a firmware update mechanism and the information model of the needed manifest.

\subsection{Architecture}
There is an Internet Draft by the SUIT group focusing on the firmware update architecture 
\parencite{suit-architecture}. This draft describes the goals and requirements of the 
architecture, although makes no mention of any particular technology. The overarching goals 
of the update process is to thwart any attempts to flash unauthorized, possibly malicious 
firmware images as well as protecting the firmware image's confidentiality. These goals 
greatly reduces the possibility of an attacker either getting control over a device or 
reverse engineering a malicious but valid firmware image as an attempt to mount an attack.



\subsection{Information Model}

\section{Contiki-NG}

\section{PKI?}

\end{document}