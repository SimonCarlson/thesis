 \documentclass[0-thesis.tex]{subfiles}

\begin{document}

\section{Network Stack}
\subsection{CoAP}
% Mention blocks and observe options

\subsection{DTLS}

\section{Hardware/Firefly}

\section{SUIT}
The IETF SUIT (Software Updates for Internet of Things) working group aims to define a firmware 
update solution that is interoperable and non-proprietary \parencite{suit}. The solution shall 
be usable on Class 1 devices as defined in RFC 7228. These devices feature ~10 KiB of RAM and 
~100 KiB code size, which makes it suitable for this thesis \parencite{rfc7228}. The solution 
may be applied to more powerful devices. The working group does not however try to define 
new transport or discovery mechanism, making their proposal angostic of any particular 
technology. As of writing, they have published two documents concerning the architecture and 
information model of a firmware update mechanism.

\subsection{Architecture}
There is an Internet Draft by the SUIT group focusing on the firmware update architecture 
\parencite{suit-architecture}. This draft describes the goals and requirements of the 
architecture, although makes no mention of any particular technology. The overarching goals 
of the update process is to thwart any attempts to flash unauthorized, possibly malicious 
firmware images as well as protecting the firmware image's confidentiality. These goals 
reduces the possibility of an attacker either getting control over a device or 
reverse engineering a malicious but valid firmware image as an attempt to mount an attack.

In order to accept an image and update itself, a device must be presented with certain 
information. Several decisions must be made before updating and the information comes in 
form of a manifest. The next section will describe the requirements posted upon this 
manifest in more detail. This manifest helps the device to make important decisions such as 
if it trusts the author of the new image, if the image is intact, if the image is applicable,
where the image should be stored and so on. This in turns means the device also has to trust 
the manifest itself, and that both manifest and update image must be distributed in a safe and 
trusted architecture. The draft \parencite{suit-architecture} presents ten qualitative 
requirements this architecture should have:
\begin{itemize}
    \item Agnostic to how firmware images are distributed:\\
            As this thesis aims to implement a prototype of an update mechanism, 
            some choices about technology has to be done. This will realistically mean 
            only a subset of the SUIT standard will be implemented as certain parts of the 
            standard is not applicable. The proposed network stack uses UDP, DTLS, and CoAP 
            for transportation and the target devices are Firefly devices running Contiki-NG.
    \item Friendly to broadcast delivery:\\
            Broadcasting not be of main concern in this thesis.
    \item Use state-of-the-art security mechanisms:\\
            The SUIT standard assumes a PKI is in place. RISE has previously developed a PKI
            suitable for IoT, this PKI is an underlying assumption for the thesis. The PKI 
            will allow for signing of the update manifest and firmware image.
    \item Rollback attacks must be prevented:\\
            The manifest will contain metadata such as monotonically increasing sequence 
            numbers and best-before timestamps to avoid rollback attacks.
    \item High reliability:\\
            This is an implementation requirement and depends heavily on the hardware of the 
            target device.
    \item Operate with a small bootloader:\\
            This is also an implementation requirement.
    \item Small parser:\\
            It must be easy to parse the fields of the update manifest as large parser can 
            get quite complex. The validation of the manifest will happen on the constrained 
            devices which further motivates a small parser and thus less complex manifests.
    \item Minimal impact on existing firmware formats:\\
            The update mechanism itself must not require any changes to existing firmware 
            formats.
    \item Robust permissions:\\
            This requirement is directed towards the administration of firmware updates and 
            how different roles interact with the devices. The thesis will not consider any 
            infrastructure outside of transporting manifest and image, as well as applying 
            the update, but will consider authorisation of parties through techniques like 
            signing.
    \item Operating modes:\\
            The draft presents three broad modes of updates: client-initiated updates, 
            server-initiated updates, and hybrid updates, where hybrids are mechanisms 
            that require interaction between the device and firmware provider before 
            updating. The thesis will look into all three of these broad classes.
\end{itemize}

An example architecture encompassing a device, a firmware update author, a firmware server,
a network operator and a device operator is presented in the draft. The author and the device 
ineract with the firmware server in order to communicate firmware updates and possibly 
manifests. This communication of the device as well as the updating itself is the concern 
of this thesis. The draft also presents device management for the device operator so that 
it is possible to track the state of updates in a network. Device management is considered 
out of scope for this thesis.

The distribution of manifest and firmware image is also discussed, with a couple of options 
being possible. The manifest and image can be distributed together to a firmware server. 
The device then recieves the manifest either via pulling or pushing and can then download
the image. Alternatively, the manifest itself can be directly sent to the device without a 
need of a firmware server, while the firmware image is put on the firmware server. After the 
device has receieved the lone manifest through some method, the firmware can be downloaded 
from the firmware server. The SUIT architecture does not enforce a specific method to be used 
when delivering the manifest and firmware, but states that an update mechanism must support 
both types.

% TODO: Images showing the differences of distribution

\subsection{Information Model}

\section{Contiki-NG}

\section{PKI?}

\end{document}