\documentclass[0-thesis.tex]{subfiles}

\begin{document}
% TODO: Dedicate this to transportation CoAP -> DTLS -> UDP
% TODO: Also format of manifest
% TODO: Encoding? CBOR?

\section{Manifest Format}
\label{sec:manifest-format}
% General intro to the chapter
% Maybe state assumptions about storage and computational capability of target devices?
As the target devices of the update mechanism are contrained IoT devices, the manifest
format must be designed with careful consideration. The format must be easy to parse in
order to reduce power consumption on devices and be small so that transportation of the
manifest is done as quickly as possible. %As a rule of thumb, each byte sent expends as
%much energy as roughly 8000 CPU cycles, so reducing the size of the manifest is of
%importance \parencite{interconnecting}.

Subsection \ref{ssec:mandatory-elements} explains the structure of the manifest. Optional
elements and their structure are explained in subsection \ref{ssec:options}. 

\subsection{Mandatory Elements}
\label{ssec:mandatory-elements}
% Show header format, motivate sizes
Different versions of manifests can feature different fields as an update mechanism
evolves, therefore the device needs to know what to expect from the manifest. This can be
encoded in a \textbf{manifest version ID}. Rollback attacks need to be prevented so that
old, vulnerable images cannot be applied. They can be mitigated through a \textbf{sequence
number}. Furthermore the \textbf{format} (ELF, binary, etc), the \textbf{size} of the
image, and a \textbf{URI} of where to find the image must be included in three separate
fields. \textbf{Vendor and class IDs}, possibly \textbf{device ID}, must also be included
so that the device knows the image is applicable and will work. Lastly, a \textbf{digest}
of the image must be included so the device can be sure the image has not been tampered
with during transport. These elements are deemed absolutely necessary and form the very
minimum upon which an update mechanism can operate. 

In addition to these elements, there are several elements that are not as critical. If a
device can update the OS and code separately or has different means of storage, a
\textbf{storage location} element must inform which storage location the update is for.
This might mean \textbf{several digests} are carried at once, which needs to be indexed by
\textbf{XIP addresses}. If the update is dependant on previous versions being present, or
a differential update needs certain other images to function, those \textbf{dependencies
and precursor images} must be described together with URIs of where to obtain them.
Cryptographic keys can be distributed in different ways, and a \textbf{key distribution
method} element can describe which method is chosen for this update. These elements have
their use cases but as the thesis will primarily look at updating a singular MCU with one
monolithic update using a PKI storage location, precursors, and key distribution method
are not needed. Despite this it is wise to include them in the format of the manifest as
it is supposed to be interoperable, and developing a highly specific mechanism for one
particular platform and use case would go against what SUIT is trying to achieve.

The proposed format of the manifest is presented in Figure \ref{fig:manifest-format}.
Options and their format is explained in the next subsection. Many of these elements are
singular values but some elements are nested structures. Digests, dependencies, and
precursors consist of URIs mapped to image digests in order to provide a source for images
as well as validating them. Conditions map conditions type to their UUIDs. The UUIDs will
be implemented in UUID5 as suggested in \parencite{suit-information-model}, where the
vendor ID uses a vendor's registered namespace, class ID uses the vendor ID, and device ID
the class ID. This ensures ID conditions are unique among vendors, and unique among
classes and devices within vendors.

\begin{figure}
    

    \caption{The proposed manifest format.}
    \label{fig:manifest-format}
\end{figure}

\subsection{Options}
\label{ssec:options}
% Show options format, motivate sizes
The elements presented by SUIT as optional namely directives, aliases, best-before
timestamp, component identifier, weighted URIs, and processing steps will be considered
options in the manifest format. These elements should usually not be needed and preparing
for their regular occurence in the manifest would waste space. There is an argument to be
made about having the best-before timestamp as a regular element, but it is only relevant
in contexts where firmware expires which is not always the case. For rollbacks, sequence
numbers (which themselves can be timestamps) already solve that. These elements can be
included as options, mapping option codes to values, as presented in Table
\ref{tab:option-codes}. By including these as options, the manifest size can be reduced.

\begin{longtable}[]{@{}ll@{}}
    \caption{The manifest options.}
    \label{tab:option-codes}\\
    \toprule
    Option Number & Option Value\tabularnewline
    \midrule
    \endhead
    1 & directives\tabularnewline
    2 & processingSteps\tabularnewline
    3 & URIs\tabularnewline
    4 & aliases\tabularnewline
    5 & component\tabularnewline
    6 & best-before timestamp\tabularnewline
    \bottomrule
\end{longtable}

The option field in the manifest format will contain a list of option structures, each
structure consisting of an option ID, option length, and option value. The directives and
processingSteps options are structures mapping types of directives and steps to values.
URIs is a list of alternative mirrors for the image, aliases map URIs to digests just like
digests, conditions, and precursors do in the mandatory fields. Finally, component is a
singular value describing which component in a heterogeneous system is to be updated.

\begin{figure}
    \begin{bytefield}{32}
        \bitbox{16}{Num} & \bitbox{16}{Length}\\
        \bitbox{32}{Extra length $\dots$}\\
        \bitbox{32}{Value $\dots$}\\
    \end{bytefield}
    \caption{The option bit format.}
    \label{fig:option-format}
\end{figure}

Figure \ref{fig:option-format} shows the bit format for the options. Almost all options
can vary greatly in size depending on the use case and thus the length field can be
extended. The length of the value field is equal to the sum of the length field(s)
expressed in bytes.

\end{document}