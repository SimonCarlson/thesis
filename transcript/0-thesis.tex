\documentclass{kththesis}

\usepackage{csquotes} % Recommended by biblatex
\usepackage[backend=biber, sorting=none]{biblatex}
\usepackage{subfiles} % To segment the thesis
\usepackage{longtable} % For tables generated by pandoc from markdown
\usepackage{booktabs} % For rules in tables 
\usepackage{bytefield} % For protocol headers
\usepackage{listings} % For code snippets 
\usepackage{tikz} % For drawing figures
\usetikzlibrary{arrows, automata, positioning, decorations.text}
\usepackage{pgf} % For PDF graphics
\usepackage[acronym,nonumberlist,nomain,nopostdot]{glossaries} % For the glossary
\usepackage{enumitem} % For changing enumerations

\makeglossaries
\newacronym{iot}{IoT}{Internet of Things}
\newacronym{suit}{SUIT}{Software Updates for Internet of Things}
\newacronym{ietf}{IETF}{Internet Engineering Task Force}
\newacronym{ip}{IP}{Internet Protocol}
\newacronym{tcp}{TCP}{Transmission Control Protocol}
\newacronym{udp}{UDP}{User Datagram Protocol}
\newacronym{tls}{TLS}{Transport Layer Security}
\newacronym{dtls}{DTLS}{Datagram Transport Layer Security}
\newacronym{http}{HTTP}{Hypertext Transfer Protocol}
\newacronym{coap}{CoAP}{Constrained Application Protocol}
\newacronym{est}{EST}{Enrollment over Secure Transport}
\newacronym{pki}{PKI}{Public Key Infrastructure}
\newacronym{uri}{URI}{Uniform Resource Identifier}
\newacronym{sha}{SHA}{Secure Hash Algorithm}
\newacronym{mcu}{MCU}{Microcontroller Unit}
\newacronym{json}{JSON}{JavaScript Object Notation}
\newacronym{cbor}{CBOR}{Concise Binary Object Representation}
\newacronym{ace}{ACE}{Authentication and Authorization for Constrained Environments}
\newacronym{ca}{CA}{Certificate Authority}

\addbibresource{references.bib} % The file containing our references, in BibTeX format

\definecolor{background}{HTML}{EEEEEE}

\lstdefinelanguage{blockc}{
    language=C,
    basicstyle=\ttfamily,
    numbers=left,
    numberstyle=\scriptsize,
    stepnumber=1,
    numbersep=8pt,
    %showstringspaces=false,
    breaklines=true
    %frame=lines,
    %backgroundcolor=\color{background},
}

\lstdefinelanguage{blockjson}{
    basicstyle=\ttfamily,
    numbers=left,
    numberstyle=\scriptsize,
    stepnumber=1,
    numbersep=8pt
}

\providecommand{\keywords}[1]{\textbf{\textit{Keywords:}} #1}


% TODO: Remove WIP
\title{Software and Firmware Updates for Internet of Things (WIP)}
\alttitle{Mjukvaru- och Firmwareuppdateringar för Internet of Things}
\author{Simon Carlson}
\email{scarlso@kth.se}
\supervisor{Farhad Abtahi}
\examiner{Elena Dubrova}
\programme{Master in Information Technology}
\school{School of Computer Science and Communication}
\date{\today}


\begin{document}

% Frontmatter includes the titlepage, abstracts and table-of-contents
\frontmatter

\titlepage

\begin{abstract}
\textbf{E}\\


\noindent\keywords{x, y, z}
\end{abstract}

\begin{otherlanguage}{swedish}
    \begin{abstract}
        Svensk sammanfattning här.
    \end{abstract}
\end{otherlanguage}


\tableofcontents
\listoftables
\listoffigures
\lstlistoflistings
\printglossaries

% Mainmatter is where the actual contents of the thesis goes
\mainmatter

% We use the \emph{biblatex} package to handle our references.  We therefore use the
% command \texttt{parencite} to get a reference in parenthesis, like this
% \parencite{heisenberg2015}.  It is also possible to include the author as part of the
% sentence using \texttt{textcite}, like talking about the work of
% \textcite{einstein2016}.

\chapter{Introduction}
\subfile{1-introduction}

\chapter{Background}
\subfile{2-background}

\chapter{Update Mechanism Architecture}
\subfile{3-architecture}

\chapter{Prototype Implementation}
\subfile{4-prototype-implementation}

\chapter{Evaluation and Results}
\subfile{5-evaluation-and-results}

\chapter{Discussion}
\subfile{6-discussion}

\printbibliography[heading=bibintoc] % Print the bibliography (and make it appear in the table of contents)

% As of now, placeholder
%\appendix

%\chapter{Appendix Title}

\end{document}
