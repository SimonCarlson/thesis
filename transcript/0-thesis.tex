\documentclass{kththesis}

\usepackage{csquotes} % Recommended by biblatex
\usepackage[backend=biber, sorting=none, citestyle=ieee]{biblatex}
\usepackage{subfiles} % To segment the thesis
\usepackage{longtable} % For tables generated by pandoc from markdown
\usepackage{booktabs} % For rules in tables 
\usepackage{bytefield} % For protocol headers
\usepackage{listings} % For code snippets 
\usepackage{tikz} % For drawing figures
\usetikzlibrary{arrows, automata, positioning, decorations.text}
\usepackage{pgf} % For PDF graphics
\usepackage[acronym,nonumberlist,nomain,nopostdot]{glossaries} % For the glossary
\usepackage{enumitem} % For changing enumerations
\usepackage{microtype} % Long words and line breaking
\usepackage{hyperref} % For links in table of contents
\usepackage{color} % For coloring links
\usepackage[font=small]{caption} % Smaller captions
\usepackage{pdfpages}

\makeglossaries
\newacronym{iot}{IoT}{Internet of Things}
\newacronym{suit}{SUIT}{Software Updates for Internet of Things}
\newacronym{ietf}{IETF}{Internet Engineering Task Force}
\newacronym{ip}{IP}{Internet Protocol}
\newacronym{tcp}{TCP}{Transmission Control Protocol}
\newacronym{udp}{UDP}{User Datagram Protocol}
\newacronym{tls}{TLS}{Transport Layer Security}
\newacronym{dtls}{DTLS}{Datagram Transport Layer Security}
\newacronym{http}{HTTP}{Hypertext Transfer Protocol}
\newacronym{coap}{CoAP}{Constrained Application Protocol}
\newacronym{est}{EST}{Enrollment over Secure Transport}
\newacronym{pki}{PKI}{Public Key Infrastructure}
\newacronym{uri}{URI}{Uniform Resource Identifier}
\newacronym{sha}{SHA}{Secure Hash Algorithm}
\newacronym{mcu}{MCU}{Microcontroller Unit}
\newacronym{json}{JSON}{JavaScript Object Notation}
\newacronym{cbor}{CBOR}{Concise Binary Object Representation}
\newacronym{ace}{ACE}{Authentication and Authorization for Constrained Environments}
\newacronym{ca}{CA}{Certificate Authority}
\newacronym{cose}{COSE}{CBOR Object Signing and Encryption}
\newacronym{oscore}{OSCORE}{Object Security for Constrained RESTful Environments}

\addbibresource{references.bib} % The file containing our references, in BibTeX format

\definecolor{background}{HTML}{EEEEEE}

\lstdefinelanguage{blockc}{
    language=C,
    basicstyle=\ttfamily,
    numbers=left,
    numberstyle=\scriptsize,
    stepnumber=1,
    numbersep=8pt,
    %showstringspaces=false,
    breaklines=true
    %frame=lines,
    %backgroundcolor=\color{background},
}

\lstdefinelanguage{blockjson}{
    basicstyle=\ttfamily,
    numbers=left,
    numberstyle=\scriptsize,
    stepnumber=1,
    numbersep=8pt
}

\lstdefinelanguage{manifest} {
    alsoletter={\alphabet},
    aboveskip=20pt,
    belowskip=20pt,
    %backgroundcolor=\color{seashell},
    basicstyle=\footnotesize\ttfamily,
    captionpos=b,
    breaklines=true,
    postbreak=\mbox{\textcolor{red}{$\hookrightarrow$}\space},
    numbers=left,
    stepnumber=1,
    literate={92f84}{92f84\allowbreak}{5}
    {fa227}{fa227\allowbreak}{5}
}

\providecommand{\keywords}[1]{\textbf{\textit{Keywords:}} #1}
\providecommand{\nyckelord}[1]{\textbf{\textit{Nyckelord:}} #1}

\hypersetup{
    colorlinks=true,
    linktoc=all,
    allcolors=black
}

\title{An Internet of Things Software and Firmware Update Architecture Based\\on the SUIT Specification}
\alttitle{En Mjukvaru- och Firmwareuppdateringsarkitektur för Internet of Things Baserad på SUIT-specifikationen}
\author{Simon Carlson}
\email{scarlso@kth.se}
\kthsupervisor{Farhad Abtahi}
\risesupervisor{Shahid Raza}
\examiner{Elena Dubrova}
\programme{Master in Embedded Systems}
\school{School of Computer Science and Communication}
\date{\today}


\begin{document}
\includepdf{cover-1.pdf}
% Frontmatter includes the titlepage, abstracts and table-of-contents
\frontmatter

\titlepage

\begin{abstract}
As society becomes more digitalized, cyberattacks are increasingly common and severe.
Security in the Internet of Things (IoT) is essential, and IoT devices must be updated to
patch vulnerabilities. The thesis aims to investigate the question "How can the Software
Updates for Internet of Things (SUIT) specification be applied to develop a
technology-agnostic and interoperable update architecture for heterogeneous networks of
Internet of Things devices?" The thesis project studied the SUIT specifications to gain an
understanding of what such an architecture must provide. Five high-level domains were
identified and further discussed: 1) roles of devices, servers, and operators, 2) key
management, 3) device profiles, 4) authorization, and 5) update handling. The architecture
was shown to fulfill the requirements SUIT imposes on the architecture and information
model, while being flexible and extensible. A prototype was developed in the Contiki-NG
operating system to evaluate the feasibility of the architecture. The thesis found that
applying the proposed architecture to constrained systems is feasible and would enable
updates in heterogeneous IoT networks.\\

\noindent\keywords{IoT, industrial IoT, security, Contiki-NG, embedded systems, 
                    software updates}
\end{abstract}

\begin{otherlanguage}{swedish}
    \begin{abstract}
        I takt med att samhället blir digitaliserat blir digitala attacker vanligare och
        får ökade konsekvenser. Säkerhet inom Internet of Things (IoT) är kritiskt och
        IoT-enheter måste kunna uppdateras för att laga sårbarheter. Denna uppsats ämnar
        att undersöka frågan "Hur kan Software Updates for Internet of Things
        (SUIT)-specifikationen appliceras för att utveckla en teknologiskt agnostisk och
        kompatibel uppdateringsarkitektur för heterogena nätverk av Internet of
        Things-enheter?" Uppsatsen studerade SUIT-specifikationen för att förstå vad en
        sådan arkitektur måste erbjuda. Fem abstrakta domänområden identifierades och
        diskuterades: 1) roller för enheter, uppdateringsservrar, och operatörer, 2)
        nyckelhantering, 3) enhetsprofiler, 4) auktorisering, och 5) lokal
        uppdateringshantering. Arkitekturen visades uppfylla de krav SUIT ställer på en
        arkitektur och informationsmodell samt var flexibel och kunde utökas. En prototyp
        utvecklades i Contiki-NG operativsystemet för att utvärdera genomförbarheten hos
        arkitekturen. Uppsatsen fann att det är rimligt att applicera den föreslagna
        arkitekturen på resursbegränsade enheter, vilket skulle möjliggöra uppdateringar
        för heterogena IoT-nätverk.\\

        \noindent\nyckelord{IoT, industriell IoT, säkerhet, Contiki-NG, inbyggda system,
                            mjukvaruuppdateringar}
    \end{abstract}
\end{otherlanguage}


\renewcommand{\abstractname}{Acknowledgements}
\begin{abstract}
    I'd like to thank my supervisors Shahid Raza and Farhad Abtahi, my examiner Elena
    Dubrova, and my opponent Samuel Lindemer for aiding me in finishing this work.
    Furthermore, I want to thank the many employees of RISE assisting me throughout the
    thesis, but will not mention any names in fear of missing some. Lastly, I want to
    thank my family and girlfriend for supporting me throughout my studies.
\end{abstract}

\tableofcontents
\listoftables
\listoffigures
\lstlistoflistings
\printglossaries

% Mainmatter is where the actual contents of the thesis goes
\mainmatter

% We use the \emph{biblatex} package to handle our references.  We therefore use the
% command \texttt{parencite} to get a reference in parenthesis, like this
% \parencite{heisenberg2015}.  It is also possible to include the author as part of the
% sentence using \texttt{textcite}, like talking about the work of
% \textcite{einstein2016}.

\chapter{Introduction}
\subfile{1-introduction}

\chapter{Background}
\subfile{2-background}

\chapter{Method}
\subfile{3-method}

\chapter{Results}
\subfile{4-results}

\chapter{Discussion}
\subfile{5-discussion}

\printbibliography[heading=bibintoc] % Print the bibliography (and make it appear in the table of contents)

% As of now, placeholder
\appendix

\chapter{Example Manifest}
\label{app:manifest}
The example manifest used during development and evaluation. It was generated by invoking
\texttt{./manifest.py 500-blocks-manifest.json -i 500-blocks-data.txt -v test -c test -u
update/image -m 1}. This command creates an output file named
\texttt{500-blocks-manifest.json} using the data file \texttt{500-blocks-data.txt},
manifest version 1, and namespaces \texttt{test} for both vendor and class ID. The data
file contains the server generated data for 500 blocks.

\lstinputlisting[language=manifest, caption={The example manifest \texttt{500-blocks-manifest.json} used in the thesis, pretty-printed.}]{500-blocks-manifest.json}

\chapter{Repositories}
\label{app:repos}
Repository containing client and server update code, located in \texttt{examples/suitup}:
\url{https://github.com/SimonCarlson/contiki-ng/tree/oscore_12}\\

\noindent Repository for manifest generation:
\url{https://github.com/SimonCarlson/manifest-generator}

\chapter{Bare Bones Example Source Code}
\label{app:bare-bones}
\begin{lstlisting}[language=manifest, caption={Bare bones Contiki-NG example.}, label=lst:bare-bones]
    #include "contiki.h"
    #include "contiki-net.h"
    #include "coap-engine.h"
    #include "coap-blocking-api.h"
    #include "coap-keystore-simple.h"

    PROCESS(bare_bones_process, "Bare bones process");
    AUTOSTART_PROCESSES(&bare_bones_process);

    PROCESS_THREAD(bare_bones_process, ev, data) {
    PROCESS_BEGIN();

    PROCESS_END();
    }
\end{lstlisting}

%\chapter{Standard Deviations}
%Table~\ref{tab:deviations} shows the standard deviations of measurements in figures
%\ref{fig:client-operations-energy}-\ref{fig:server-image-energy}.

%\begin{table}[h]
%\small
%\begin{tabular}{l | c c c c}
%    Measurement & CPU & LPM & TX & RX\\
%    \hline\hline
%    Client operations &&&&\\
%    \hline
%    Registration & 0.00004632148074 & 0.0000716225263 & 0.00006111165061 & 0.002411339236\\
%    Manifest & 0.00004632148074 & 0.0001535282166 & 0.0004054953317 & 0.00516864187\\
%    Decode and parse & 0.0000003631596218 & 0 & 0 & 0.0000006940594881\\
%    \hline\hline
%    Server operations &&&&\\
%    \hline
%    Register & 0.00002335727481 & 0 & 0 & 0.00003626876241\\
%    Manifest & 0.00004097692291 & 0.0001278287866 & 0.000003716510666 & 0.004300987928\\
%    \hline\hline
%    Client image transfer &&&&\\
%    \hline
%    500 blocks & 0.000918334576 & 0.001208184762 & 0.001205785423 & 0.0405420135\\
%    2000 blocks & 0.001986588496 & 0.001536797951 & 0.003752072953 & 0.05134885381\\
%    4000 blocks & 0.002718812919 & 0.003600616008 & 0.001893914224 & 0.1207698144\\
%    \hline\hline
%    Server image transfer &&&&\\
%    \hline
%    500 blocks & 0.0004936030601 & 0.001204101027 & 0.001204101027 & 0.04004990673\\
%    2000 blocks & 0.001653833922 & 0.001563795897 & 0.000868800205 & 0.05281899214\\
%    4000 blocks & 0.002960744548 & 0.003574288841 & 0.0009382682704 & 0.1207197374\\
%\end{tabular}
%\end{table}

\includepdf{cover-2.pdf}
\end{document}