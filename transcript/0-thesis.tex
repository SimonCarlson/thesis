\documentclass{kththesis}

\usepackage{csquotes} % Recommended by biblatex
\usepackage[backend=biber, sorting=none, citestyle=ieee]{biblatex}
\usepackage{subfiles} % To segment the thesis
\usepackage{longtable} % For tables generated by pandoc from markdown
\usepackage{booktabs} % For rules in tables 
\usepackage{bytefield} % For protocol headers
\usepackage{listings} % For code snippets 
\usepackage{tikz} % For drawing figures
\usetikzlibrary{arrows, automata, positioning, decorations.text}
\usepackage{pgf} % For PDF graphics
\usepackage[acronym,nonumberlist,nomain,nopostdot]{glossaries} % For the glossary
\usepackage{enumitem} % For changing enumerations
\usepackage{microtype} % Long words and line breaking
\usepackage{hyperref} % For links in table of contents
\usepackage{color} % For coloring links
\usepackage[font=normalsize]{caption} % To preserve caption size when shrinking table font size

\makeglossaries
\newacronym{iot}{IoT}{Internet of Things}
\newacronym{suit}{SUIT}{Software Updates for Internet of Things}
\newacronym{ietf}{IETF}{Internet Engineering Task Force}
\newacronym{ip}{IP}{Internet Protocol}
\newacronym{tcp}{TCP}{Transmission Control Protocol}
\newacronym{udp}{UDP}{User Datagram Protocol}
\newacronym{tls}{TLS}{Transport Layer Security}
\newacronym{dtls}{DTLS}{Datagram Transport Layer Security}
\newacronym{http}{HTTP}{Hypertext Transfer Protocol}
\newacronym{coap}{CoAP}{Constrained Application Protocol}
\newacronym{est}{EST}{Enrollment over Secure Transport}
\newacronym{pki}{PKI}{Public Key Infrastructure}
\newacronym{uri}{URI}{Uniform Resource Identifier}
\newacronym{sha}{SHA}{Secure Hash Algorithm}
\newacronym{mcu}{MCU}{Microcontroller Unit}
\newacronym{json}{JSON}{JavaScript Object Notation}
\newacronym{cbor}{CBOR}{Concise Binary Object Representation}
\newacronym{ace}{ACE}{Authentication and Authorization for Constrained Environments}
\newacronym{ca}{CA}{Certificate Authority}
\newacronym{cose}{COSE}{CBOR Object Signing and Encryption}
\newacronym{oscore}{OSCORE}{Object Security for Constrained RESTful Environments}

\addbibresource{references.bib} % The file containing our references, in BibTeX format

\definecolor{background}{HTML}{EEEEEE}

\lstdefinelanguage{blockc}{
    language=C,
    basicstyle=\ttfamily,
    numbers=left,
    numberstyle=\scriptsize,
    stepnumber=1,
    numbersep=8pt,
    %showstringspaces=false,
    breaklines=true
    %frame=lines,
    %backgroundcolor=\color{background},
}

\lstdefinelanguage{blockjson}{
    basicstyle=\ttfamily,
    numbers=left,
    numberstyle=\scriptsize,
    stepnumber=1,
    numbersep=8pt
}

\lstdefinelanguage{manifest} {
    alsoletter={\alphabet},
    aboveskip=20pt,
    belowskip=20pt,
    %backgroundcolor=\color{seashell},
    basicstyle=\footnotesize\ttfamily,
    captionpos=b,
    breaklines=true,
    postbreak=\mbox{\textcolor{red}{$\hookrightarrow$}\space},
    numbers=left,
    stepnumber=1,
    literate={92f84}{92f84\allowbreak}{5}
    {fa227}{fa227\allowbreak}{5}
}

\providecommand{\keywords}[1]{\textbf{\textit{Keywords:}} #1}
\providecommand{\nyckelord}[1]{\textbf{\textit{Nyckelord:}} #1}

\hypersetup{
    colorlinks=true,
    linktoc=all,
    allcolors=black
}

% TODO: Remove WIP
\title{An Internet of Things Software and Firmware Update Architecture Based\\on the SUIT specification (WIP)}
\alttitle{En Mjukvaru- och Firmwareuppdateringsarkitektur för Internet of Things Baserad på SUIT-specifikationen (WIP)}
\author{Simon Carlson}
\email{scarlso@kth.se}
\kthsupervisor{Farhad Abtahi}
\risesupervisor{Shahid Raza}
\examiner{Elena Dubrova}
\programme{Master in Information Technology}
\school{School of Computer Science and Communication}
\date{\today}


\begin{document}
% Frontmatter includes the titlepage, abstracts and table-of-contents
\frontmatter

\titlepage

\begin{abstract}
In recent years the general public has become increasingly aware of digital attacks and
intrusions affecting their day to day lives as society becomes more connected. Security in
Internet of Things (IoT) is a essential and as IoT is expected to grow rapidly insecure
devices can cause even more problems in the future. There is a need for secure software
and firmware updates for IoT devices as vulnerabilities must be patched. For many IoT
devices this mean patches must be applied over long distances and possibly unreliable
communication channels as sending a technician to each and every device is unfeasible. 

The thesis aims to investigate the question "How can the SUIT specification be applied to
develop a technology agnostic and interoperable update architecture for heterogeneous
networks of Internet of Things devices?". The thesis project studied the SUIT
specifications to gain an understanding of what such an architecture must provide. Five
high-level domains were identified and further on defined: roles of devices, servers, and
operators, key management, device profiles, authorization, and update handling. Two
profiles that instantiate the proposed architecture were defined and one of them
implemented as a prototype. 

The architecture was shown to fulfill the requirements SUIT poses on the architecture and
information model while being flexible and extensible. The prototype was measured for
energy consumption, code size, and communication overhead. The client occupied 91343 bytes
and server 91643 bytes on a Firefly device with 512 KB flash. Transferring image data was
the largest source of energy consumption during an update procedure where the radio in
receive mode accounted for up to 83\% of the total image transfer energy consumed in the
client and 81\% on the server. The thesis found that applying the proposed architecture to
constrained systems is feasible and would enable updates in heterogeneous IoT networks
while maintaining the minimum level of security sought for, with the flexibility of
creating a more rigorous system within the bounds of the architecture.\\

\noindent\keywords{IoT, industrial IoT, security, Contiki-NG, embedded systems, 
                    software updates}
\end{abstract}

\begin{otherlanguage}{swedish}
    \begin{abstract}
        De senaste åren har allmänheten blivit mer informerad kring digitala attacker och
        intrång som påverkar deras vardagliga liv då samhället blir mer och mer
        uppkopplat. Säkerhet inom Internet of Things (IoT) är kritiskt och eftersom IoT
        förväntas växa hastigt de kommande åren kan osäkrade enheter orsaka ännu större
        skada i framtiden. Det finns ett behov för säkra mjukvaru- och
        firmwareuppdateringar för IoT-enheter då sårbarheter måste lagas. För många
        IoT-enheter innebär detta att uppdateringar måste appliceras över långa avstånd
        och möjligtvis opålitliga nätverk då det vore orimligt att skicka en tekniker till
        var och varje enhet. 
        
        Denna uppsats ämnar att undersöka frågan "Hur kan SUIT-specifikationen appliceras
        för att utveckla en teknologiskt agnostisk och kompatibel uppdateringsarkitektur
        för heterogena nätverk av Internet of Things-enheter?". Uppsatsen studerade
        SUIT-specifikationen för att förstå vad en sådan arkitektur måste erbjuda. Fem
        abstrakta domänområden identifierades och definierades: roller för enheter,
        updateringsservrar, och operatörer, nyckelhantering, enhetsprofiler,
        auktorisering, och lokal uppdateringshantering. Två profiler som instansierar den
        föreslagna arkitekturen definierades och en av dem implementerades som en
        prototyp. 
        
        Arkitekturen visades uppfylla de krav SUIT ställer på en arkitektur och
        informationsmodell samt var flexibel och kunde utökas. Prototypen utvärderades för
        energikonsumption, kodstorlek, och kommunikations-overhead. Klienten använde sig
        av 91343 bytes och servern av 91643 bytes på en Firefly-enhet med 512 KB flash.
        Överföringen av avbildningar var den största källan av energikonsumption under en
        uppdateringsprocedur där den mottagande radion stod för upp till 83\% av den
        totala energin i klienten och up till 81\% på servern. Uppsatsen fann att det är
        rimligt att applicera den föreslagna arkitekturen på resursbegränsade enheter. Det
        skulle göra uppdateringar för heterogena IoT-nätverk möjliga samtidigt som den
        minsta acceptabla säkerhetsnivån upprätthålls, med möjligheter att skapa ett mer
        rigoröst system inom ramarna för arkitekturen.\\

        \noindent\nyckelord{IoT, industriell IoT, säkerhet, Contiki-NG, inbyggda system,
                            mjukvaruuppdateringar}
    \end{abstract}
\end{otherlanguage}


\renewcommand{\abstractname}{Acknowledgements}
\begin{abstract}
    E
\end{abstract}

\tableofcontents
\listoftables
\listoffigures
\lstlistoflistings
\printglossaries

% Mainmatter is where the actual contents of the thesis goes
\mainmatter

% We use the \emph{biblatex} package to handle our references.  We therefore use the
% command \texttt{parencite} to get a reference in parenthesis, like this
% \parencite{heisenberg2015}.  It is also possible to include the author as part of the
% sentence using \texttt{textcite}, like talking about the work of
% \textcite{einstein2016}.

\chapter{Introduction}
\subfile{1-introduction}

\chapter{Background}
\subfile{2-background}

\chapter{Method}
\subfile{3-method}

\chapter{Results}
\subfile{4-results}

\chapter{Discussion}
\subfile{5-discussion}

\printbibliography[heading=bibintoc] % Print the bibliography (and make it appear in the table of contents)

% As of now, placeholder
\appendix

\chapter{Example Manifest}
\label{app:manifest}
The example manifest used during development and evaluation. It was generated by invoking
\texttt{./manifest.py 500-blocks-manifest.json -i 500-blocks-data.txt -v test -c test -u
update/image -m 1}. This command creates an output file named
\texttt{500-blocks-manifest.json} using the data file \texttt{500-blocks-data.txt},
manifest version 1, and namespaces \texttt{test} for both vendor and class id. The data
file contains the server generated data for 500 blocks.

\lstinputlisting[language=manifest, caption={The example manifest \texttt{500-blocks-manifest.json} used in the thesis, pretty-printed.}]{500-blocks-manifest.json}

\chapter{Repositories}
\label{app:repos}
Repository containing client and server update code, located in \texttt{examples/suitup}:
\url{https://github.com/SimonCarlson/contiki-ng/tree/oscore_12}\\

\noindent Repository for manifest generation:
\url{https://github.com/SimonCarlson/manifest-generator}

\chapter{Bare Bones Example Source Code}
\label{app:bare-bones}
\begin{lstlisting}[language=manifest, caption={Bare bones Contiki-NG example.}, label=lst:bare-bones]
    #include "contiki.h"
    #include "contiki-net.h"
    #include "coap-engine.h"
    #include "coap-blocking-api.h"
    #include "coap-keystore-simple.h"

    PROCESS(bare_bones_process, "Bare bones process");
    AUTOSTART_PROCESSES(&bare_bones_process);

    PROCESS_THREAD(bare_bones_process, ev, data) {
    PROCESS_BEGIN();

    PROCESS_END();
    }
\end{lstlisting}


\end{document}
