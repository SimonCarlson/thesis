\documentclass{kththesis}

\usepackage{csquotes} % Recommended by biblatex
\usepackage[backend=biber, sorting=none]{biblatex}
\usepackage{subfiles} % To segment the thesis
\usepackage{longtable} % For tables generated by pandoc from markdown
\usepackage{booktabs} % For rules in tables 
\usepackage{bytefield} % For protocol headers
\usepackage{listings} % For code snippets 
\usepackage{tikz} % For drawing figures
\usetikzlibrary{arrows, automata, positioning}
\usepackage{pgf} % For PDF graphics
\addbibresource{references.bib} % The file containing our references, in BibTeX format

\title{Software and Firmware Updates for Internet of Things}
\alttitle{Mjukvaru- och Firmwareuppdateringar för Internet of Things}
\author{Simon Carlson}
\email{scarlso@kth.se}
\supervisor{Farhad Abtahi}
\examiner{Elena Dubrova}
\programme{Master in Information Technology}
\school{School of Computer Science and Communication}
\date{\today}


\begin{document}

% Frontmatter includes the titlepage, abstracts and table-of-contents
\frontmatter

\titlepage

\begin{abstract}
IoT devices are used in a variety of use cases and the use of IoT is expected to grow
significantly the coming years. Unsecured devices have the past years led to IoT devices
being leveraged in attacks or attacked themselves, which is in the face of growth
unacceptable. In order to secure devices, they need to be updated to patch existing and
future vulnerabilities. However, update mechanisms for remote and constrained IoT devices
are most often proprietary and closed source. In order to develop an open and
interoperable, the IETF has created a working group called SUIT. SUIT has developed a
standard for IoT updates, which forms the basis of this work. The thesis explains the
state of the art and SUIT standard, from which it develops a software and firmware update
mechanism for constrained IoT devices. The mechanism fulfills X and Y from the standard,
while showing G and H reliability and power requirements on the nodes.

\end{abstract}


\begin{otherlanguage}{swedish}
    \begin{abstract}
        Svensk sammanfattning här.
    \end{abstract}
\end{otherlanguage}


\tableofcontents


% Mainmatter is where the actual contents of the thesis goes
\mainmatter

% We use the \emph{biblatex} package to handle our references.  We therefore use the
% command \texttt{parencite} to get a reference in parenthesis, like this
% \parencite{heisenberg2015}.  It is also possible to include the author as part of the
% sentence using \texttt{textcite}, like talking about the work of
% \textcite{einstein2016}.

\chapter{Introduction}
\subfile{1-introduction}

\chapter{Theoretic Background}
\subfile{2-theoretic-background}

\chapter{Transportation of Firmware Images}
\subfile{3-transportation-of-firmware-images}
% Maybe join these chapters, discussing manifest and updating in the same chapter?
% If so, create a new chapter for implementation of the design?
\chapter{Updating of Firmware Images}
\subfile{4-updating-of-firmware-images}

\chapter{Evaluation and Results}
\subfile{5-evaluation-and-results}

\chapter{Discussion}
\subfile{6-discussion}

\printbibliography[heading=bibintoc] % Print the bibliography (and make it appear in the table of contents)

% As of now, placeholder
\appendix

\chapter{Appendix Title}

\end{document}
