\documentclass[0-thesis.tex]{subfiles}

\begin{document}

\section{Background}
% Explain what IoT is
Internet of Things, or IoT, is the notion of connecting physical objects to the
world-spanning Internet in order to facilitate services and communication both
machine-to-human and machine-to-machine. By having everything connected, from everyday
appliances to vehicles to critical parts of infrastructure, computing will be ubiquitous
and the Internet of Things realized. The benefits from the IoT can range from quality of
life services, such as controlling lights and thermostats from afar, to data gathering
through wireless sensor networks, to enabling life critical operations such as monitoring
a pacemaker. IoT is a broad definition and fits many different devices in many different
environment, what they all have in common is that they are physical and connected devices.

In recent years the general public has become increasingly aware of digital attacks and
intrustions affecting their day to day lives. In 2016 the DNS provider Dyn was attacked by
a botnet consisting of heterogenous IoT devices infected by the Mirai malware. Devices
such as printers and baby monitors were leveraged to launch an attack on Dyn's services
which affected sites like Airbnb, Amazon, and CNN \parencite{perlroth_2016}. Cardiac
devices implanted in patients were also discovered to be unsafe, with hackers being able
to deplete the batteries of pacemakers prematurely \parencite{hern_2017}. These cases and
many others make it clear that security in IoT is a big deal.

According to Bain \& Company, the largest barrier for Internet of Things (IoT) adoptation
is security concerns, and customers would buy an average of 70\% more IoT devices if they
were secured \parencite{ali_bosche_ford_2018}. Despite security being lacking today in
IoT, the field is expected to grow rapidly and an increase in unsecured devices could
spell disaster. Securing IoT equipment such as printers, baby monitors, and pacemakers is
imperative to prevent future attacks. But what about the devices currently employed
without security? They need to be updated and patched in order to fix these
vulnerabilities, a non-trivial task.

\section{Problem Statement}
There is a need for secure software and firmware updates for IoT devices as
vulnerabilities must be patched. For many IoT devices this mean patches must be applied
over long distances and possibly unreliable communication channels as sending a technician
to each and every device is unfeasible. There are non-open, proprietary solutions
developed for specific devices but no open and interoperable standard. The IETF SUIT
working group aims to define an architecture for such a mechanism without defining new
transport or discovery mechanisms \parencite{suit}. By expanding upon the work of the SUIT
group, a standardized update mechanism suitable for battery powered, constrained, and
remote IoT devices can be developed.

\subsection{Problem}
The thesis project will examine the architecture and information model proposed by SUIT in
order to create and evaluate an updating mechanism for battery powered, constrained, and
remote IoT devices with a life span exceeding five years. The update mechanism can be
considered as two parts, communication and upgrading. Communication will happen over
unreliable channels and the payloads must arrive intact and untampered with. The upgrade
mechanism itself must ensure integrity of the target image and safely perform the upgrade
with a limited amount of memory. The thesis will examine the update mechanism from the
viewpoint of IoT devices running the Contiki-NG operating system on 32-bit ARM Cortex M3
processors. A suitable public key infrastructure developed by RISE will also be an
underlying assumption for the work. The thesis aims to investigate the problem "How can
the SUIT architecture be used to provide secure software and firmware update for the
Internet of Things?".

\subsection{Purpose}
The purpose of the thesis project is to provide an open and interoperable software and
firmware update mechanism that complies with the standards suggested by the IETF SUIT
working group. This will aid other projects trying to secure their IoT devices following
accepted standards.

\subsection{Goal}
The goals of the thesis are to study current solutions and then propose lightweight
end-to-end protocols that can be used when updating IoT devices. The degree project shall
deliver specifications and prototypes of the protocols in an IoT testbed, as well as a
thesis report.

\section{Methodologies}
The thesis will follow a mix of qualitative and quantitative methodologies. The SUIT group
defines some goals or constraints a suitable updating mechanism should follow, but as
their proposed architecture is agnostic of any particular technology these goals cannot be
easily quantified. It is better to regard them as qualitative properties the mechanism
should have. In addition to this there are some relevant measurements, such as reliability
of the communication and memory and power requirements of the updates, that can be used in
an evaluation. The quantitative part of the evaluation will follow an objective,
experimental approach, while the qualitative part will follow a interpretative approach.

\section{Risks, Consequences, and Ethics}
IoT devices need to be securely updated in order to fix vulnerabilities and undergo
maintenance to ensure their future operability. Failing to do so could leave a device or
network unsecured, posing a security risk. Furthermore the performance of the service
provided by the IoT device may itself degrade. Lastly, if updates are not applied in a
safe and secure manner, the updating mechanism itself might cause service disruptions.
These risks are serious and could pose ethical dilemmas depending on what the IoT device
is used for.

\section{Scope}
The updating mechanism can be roughlt split into three parts: the actual updating or
flipping of images on the device, the transportation of the new image, and managing a
heterogenous network of devices needing different updates at different times. This thesis
will only look at the first two parts, updating a firmware image on the device and
transportation of the image. The thesis will not be concerned about device management.

\section{Outline}
Chapter two describes the theoretical background needed to understand the results of the
thesis. This includes the network protocols, hardware, operating system, and PKI being
used in the thesis. Chapter three describes the design of the communication protocols used
in the update mechanism. Chapter four describes how the local update mechanism works and
how devices can upgrade their images. Chapter five evaluates the developed updating
mechanism and presents the results. Chapter six ends the report with a discussion about
the update mechanism and its results.

\end{document}