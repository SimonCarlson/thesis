 \documentclass[0-thesis.tex]{subfiles}

\begin{document}
With an understanding of why IoT updates are needed, a solution must be proposed. In order
to propose a solution, however, the current state-of-the-art must be understood. IoT
networks use different protocols compared to traditional computing for reasons such as
reliability and performance. This chapter presents the background needed to understand the
solutions presented in the thesis. 

Section~\ref{sec:network} introduces the network protocols used in IoT contexts and
motivates their use over other protocols. The following section, Section~\ref{sec:suit}
presents and explains the SUIT architecture and information model and their respective
requirements as formulated by the IETF. Finally, Section~\ref{sec:contiki-ng} presents the
Contiki-NG operating system, which the updating mechanism will be developed for.

\section{IoT Network Stack}
\label{sec:network}
Network protocols in IoT networks operate under different circumstances compared to
traditional computer networks. Whereas traditional networks enjoy high reliability, high
throughput, and high computational performance, IoT networks are defined by their low
power requirements, low reliability, and low computational performance on edge devices.
This puts some constraints on the protocols used in, IoT as they must properly handle
these characteristics.

One of the most widely used network protocol stacks today in traditional networks is based
on the \gls{ip}. The stack uses \gls{tcp} as a transportation protocol, usually with
\gls{tls} for security, and a common application layer protocol is \gls{http}. TCP is,
however, poorly suited for IoT networks, as it is a connection based, stateful protocol
which tries to ensure the guaranteed delivery of packets in the correct order. There is
also advanced congestion control mechanisms in TCP which are hard to apply on unreliable
low-bandwidth networks. IoT networks often utilize \gls{udp} as a transport protocol
instead. UDP is also an IP-based protocol but is connectionless and less reliable than
TCP, performing on a best-effort level instead. Despite being less reliable, UDP often
performs better in IoT contexts.

As TLS is based on the same assumptions as TCP it is unsuitable for UDP networks. UDP
networks still need confidentiality and integrity through some means and thus use
\gls{dtls}, which is a version of TLS enhanced for use in datagram oriented protocols.
HTTP can be used over UDP for the application layer, but as HTTP is encoded in
human-readable plaintext, it is unnecessarily verbose and not optimal for constrained
networks. Instead, \gls{coap} is a common protocol for the application layer in IoT
networks. Figure~\ref{fig:stack-comparison} shows the equivalent protocols for IoT network
stacks versus traditional network stacks. 

\begin{figure}[t]
    \begin{bytefield}[bitformatting=\small, bitwidth=1.1em]{30}
        \bitbox[]{12}{IoT stack} & \bitbox[]{4}{} & \bitbox[]{12}{Traditional stack}\\
        \bitbox{12}{CoAP(s)} & \bitbox[]{4}{$\Longleftrightarrow$} \bitbox{12}{HTTP(S)}\\
        \bitbox{6}{UDP} & \bitbox{6}{DTLS} & \bitbox[]{4}{$\Longleftrightarrow$} \bitbox{6}{TCP} & \bitbox{6}{TLS}\\
        \bitbox{12}{IPv6} & \bitbox[]{4}{$\Longleftrightarrow$} \bitbox{12}{IPv4 or IPv6}\\
        \bitbox{6}{IEEE 802.15.4} & \bitbox{6}{6LoWPAN} & \bitbox[]{4}{$\Longleftrightarrow$} \bitbox{12}{IEEE 802.15.4}\\
        \bitbox{12}{IEEE 802.15.4} & \bitbox[]{4}{$\Longleftrightarrow$} \bitbox{12}{IEEE 802.15.4}\\
    \end{bytefield}
    \caption{Comparison of network stacks between IoT networks and traditional networks.}
    \label{fig:stack-comparison}
\end{figure}

In this section, Section~\ref{ssec:udp} explains UDP and why it is the preferred transport
protocol in IoT networks. Section~\ref{ssec:dtls} briefly explains TLS, why it is
unsuitable for IoT networks, the differences between TLS and DTLS and why DTLS is used
instead. Section~\ref{ssec:coap} describes the CoAP protocol,
Section~\ref{ssec:cbor-and-cose} the CBOR encoding and its security protocol COSE, and
Section~\ref{ssec:oscore} briefly introduces OSCORE. Lastly, sections \ref{ssec:est-coaps}
and \ref{ssec:ace} briefly introduces the EST-coaps protocol and ACE framework, enabling
asymmetric cryptography and authorization in an IoT context.

\subsection{User Datagram Protocol}
\label{ssec:udp}
\acrfull{udp} is a stateless and asynchronous transfer protocol for IP \parencite{rfc768}.
It does not provide any reliability mechanisms but is instead a best-effort protocol. It
also does not guarantee delivery of messages. For general purposes in unconstrained
environments TCP is usually the favored transport protocol as it is robust and reliable,
but in environments where resources are scarce and networks unreliable, a stateful
protocol like TCP could face issues. Since TCP wants to ensure packet delivery, it will
retransmit packages generating a lot of traffic and processing required for a receiver.
Also, if the connection is too unstable TCP will not work at all since it can no longer
guarantee the packets arrival. The best-effort approach of UDP is favorable in these
situations, in addition to UDP being a lightweight protocol requiring a smaller memory
footprint to implement.

\subsection{Datagram Transport Layer Security}
\label{ssec:dtls}
\acrfull{dtls} is a protocol which adds confidentiality and integrity to datagram
protocols like UDP \parencite{rfc6347}. The protocol is designed to prevent eavesdropping,
tampering, or message forgery. DTLS is based on TLS, a similar protocol for stateful
transport protocols such as TCP, which would not work well on unreliable networks, as
previously discussed. The main issues with using TLS over unreliable networks is that TLS
decryption is dependent on previous packets, meaning the decryption of a packet will fail
if the previous packet was not received. In addition, the TLS handshake procedure assumes
all handshake messages are delivered reliably, which is rarely the case in IoT networks
using UDP.

DTLS solves this by removing stream ciphers, effectively making decryption an independent
operation between packets, as well as adding explicit sequence numbers. Furthermore, DTLS
supports packet retransmission, reordering, as well as fragmenting DTLS handshake messages
into several DTLS records. These mechanisms make the handshake process feasible over
unreliable networks. 

By splitting messages into different DTLS records, fragmentation at the IP level can be
avoided since a DTLS record is guaranteed to fit an IP datagram. IP fragmentation is
problematic in low-performing networks, since if a single fragment of an IP packet is
dropped, all fragments of that packet must be retransmitted, and thus fragmenting at the IP
level should be avoided. Since DTLS is designed to correctly handle reordering and
retransmission in lossy networks, splitting messages into several DTLS records is no
problem, and if one record is lost only that record needs to be retransmitted in a single
IP packet.

In order to communicate via TLS and DTLS, a handshake has to be carried out. The handshake
establishes parameters such as protocol version, cryptographic algorithms, and shared
secrets. The TLS handshake involves hello messages for establishing algorithms, exchanging
random values, and checking for earlier sessions. Then cryptographic parameters are shared
in order to agree on a shared premaster secret. The parties authenticate each other via
public key encryption, generate a shared master secret based on the premaster secret, and
finally verify that their peer has the correct security parameters.

The DTLS handshake adds to this a stateless cookie exchange to complicate DoS attacks,
some modifications to the handshake header to make communication over UDP possible, and
retransmission timers since the communication is unreliable. Otherwise the DTLS handshake
is the same as the TLS handshake.

\subsection{Constrained Application Protocol}
\label{ssec:coap}
\acrfull{coap} is an application layer protocol designed to be used by constrained devices
over networks with low throughput and possibly high unreliability for machine-to-machine
communication \parencite{rfc7252}. While designed for constrained networks, a design
feature of CoAP is how it is easily interfaced with HTTP so that communication over
traditional networks can be proxied. CoAP uses a request/response model similar to HTTP
with method codes and request methods that are easily mapped to those of HTTP.
Furthermore, CoAP is a RESTful protocol utilizing concepts such as endpoints, resources,
and \gls{uri}. These features make it easier to design IoT applications that seamlessly
interact with traditional web services. Additionally, CoAP offers features such as
multicast support, asynchronous messages, low header overhead, and UDP and DTLS bindings
which are all suitable for constrained environments.

As CoAP is usually implemented on top of UDP, communication is stateless and asynchronous.
For this reason CoAP defines four message types: Confirmable, Non-confirmable,
Acknowledgment, and Reset. Confirmable messages must be answered with a corresponding
Acknowledgment. This provides one form of reliability over an otherwise unreliable
channel. Non-confirmable messages do not require an Acknowledgment and thus act
asynchronously. Reset messages are used when a recipient is unable to process a
Non-confirmable message.

Since CoAP is based on unreliable means of transport, there are some lightweight
reliability and congestion control mechanisms in CoAP. Message IDs allow for detection of
duplicate messages and tokens allow asynchronous requests and responses to be paired
correctly. There is also a retransmission mechanism with an exponential back-off timer for
Confirmable messages so that lost Acknowledgments do not cause a flood of retransmissions.
Additionally, CoAP features piggybacked responses, meaning a response can be sent in the
Acknowledgment of a Confirmable or Non-Confirmable request if the response fits and is
available right away. This also reduces the amount of messages sent by the protocol.

The length of the payload is dependent on the carrying protocol and is calculated
depending on the size of the CoAP header, token, and options, as well as the maximum DTLS
record size. Section 4.6 of the CoAP standard says "If the Path MTU [Maximum Transmission
Unit] is not known for a destination, an IP MTU of 1280 bytes SHOULD be assumed; if
nothing is known about the size of the headers, good upper bounds are 1152 bytes for the
message size and 1024 bytes for the payload size" \parencite{rfc7252}.

Since firmware images can be relatively large, their size needs to be handled during
transportation, which can be done via block-wise transfers \parencite{rfc7959}. A Block
option allows stateless transfer of a large file separated in different blocks. Each block
can be individually retransmitted, and by using monotonically increasing block numbers, the
blocks can be reassembled. The size of blocks can also be negotiated between server and
client, meaning they can always find a suitable block size, making the mechanism quite
flexible. Another option of interest is the observe option, which allows a client to be
notified by the server when a particular resource changes. This option can be used in a
pull or hybrid update architecture, meaning the device does not have to continuously poll
the server for a new update.

\subsection{Concise Binary Object Representation and CBOR Object Signing}
\label{ssec:cbor-and-cose}
\gls{cbor} aims to be an extensible data format providing very small code sizes
\parencite{rfc7049}. It supports simple values as well as arrays and maps, meaning it is
easy to map to and from JSON while being more compact. It can be used to encode the most
common data formats from Internet standards while being easy to parse, even for resource
constrained systems. \gls{cose} defines some basic security services for CBOR objects such
as encryption and signing \parencite{rfc8152}. COSE is also designed for use on
constrained devices, as CBOR would primarily be used on these devices.

\subsection{Object Security for Constrained RESTful Environments}
\label{ssec:oscore}
\gls{oscore} is a way of protecting CoAP messages at the application level using COSE
\parencite{oscore}. When using CoAP secured by DTLS, security is terminated at
intermediate proxies, meaning data can be eavesdropped on or modified. OSCORE avoids this
by protecting sensitive parts of CoAP messages, but does not fully protect the entire CoAP
message. The mapping capabilities between CoAP and HTTP are not affected by the use of
OSCORE, and OSCORE protected messages can be translated to use different transport
protocols as well. OSCORE is therefore well suited to use within constrained networks, and
also for interfacing with traditional networks while retaining end-to-end security. The
OSCORE specification is a work in progress.

\subsection{EST-coaps}
\label{ssec:est-coaps}
EST-coaps is a protocol being standardized for certificate enrollment in constrained
environments using CoAP \parencite{est-coaps}. By allowing IoT devices to enroll for
certificates, asymmetric encryption can be used even in a constrained environment.
EST-coaps is heavily based on \gls{est}, which was developed for traditional, less
constrained networks, and is thus incompatible with the SUIT specification, which is
specified to work on constrained devices \parencite{rfc7030}. EST-coaps retains much of
the functionality and structure of EST but modifies it slightly to work over CoAP, DTLS,
and UDP instead of HTTP, TLS, and TCP, by making use of CoAP's block requests and
responses to remedy the relatively large sizes of certificates. A corresponding EST
profile for OSCORE is in development \parencite{est-oscore}.

\subsection{Authorization and Authentication for Constrained Environments}
\label{ssec:ace}
\gls{ace} is a framework being standardized for authorization and authentication for IoT
contexts, based on the OAuth 2.0 framework \parencite{ace}. ACE allows clients in a
network to access protected resources through authorization tokens. 

A client can request an authorization token from an authorization server and then make a
request with the token to a resource server. If the token is valid and the resource
requested matches the level of authority associated with the token, access to the resource
is granted. The resource server can perform an introspection request to the authorization
server if it needs extra information to verify the token. The exchange is depicted in
Figure~\ref{fig:ace-flow}.

\begin{figure}
    \caption[The protocol flow of ACE.]
        {The protocol flow of ACE. Adapted from \parencite{ace}.}
    \label{fig:ace-flow}
    \includegraphics{images/ace.pdf}
\end{figure}

Authorization is important and distinct from identification, which can be achieved
through, for instance, a \gls{pki}. As the architecture proposed in this thesis aims to be
standardized, it should prepare for different parties assuming different roles in the
context of updating devices. Using authorization tokens is a scalable and configurable way
of achieving that.

\section{Software Updates for Internet of Things}
\label{sec:suit}
The IETF \acrfull{suit} working group aims to define a firmware update solution for IoT
devices that is interoperable and non-proprietary \parencite{suit}. The working group does
not, however, try to define new transport, discovery, or security mechanisms, making their
proposal agnostic of any particular technology. SUIT aims to define both a mechanism for
transporting firmware images as well as the meta-data needed to securely update an IoT
system.

Section~\ref{ssec:architecture} presents the SUIT architecture and
Section~\ref{ssec:information-model} presents the SUIT information model.

\subsection{Architecture}
\label{ssec:architecture}
There is an Internet-Draft by the SUIT group focusing on the architecture of an IoT update
mechanism \parencite{suit-architecture}. Internet-Drafts are works in progress and may be
altered or considered obsolete at any time. This thesis references the architecture draft
issued January 16 2019. This draft describes the goals and requirements of such an
architecture, although it makes no mention of any particular technology. The overarching
goals of the update process is to thwart any attempts to flash unauthorized, possibly
malicious firmware images, as well as protecting the firmware image's confidentiality and
integrity. These goals reduce the possibility of an attacker either getting control over
a device or reverse-engineering a malicious, but valid, firmware image as an attempt to
mount an attack.

In order to accept an image and update itself, a device must make several decisions about
the validity and suitability of the image. The information needed comes in the form of a
manifest. The next section will describe the requirements imposed on this manifest in
more detail. The manifest helps the device make important decisions, such as if it trusts
the author of the new image, if the image is intact, if the image is applicable, where the
image should be stored, and so on. This, in turn, means the device also has to trust the
manifest itself, and that both manifest and update image must be distributed in a safe and
trusted architecture. Table~\ref{tab:architecture-requirements} shows the requirements
imposed on the architecture by SUIT.

\begin{small}
\begin{longtable}[]{@{}ll@{}}
        \caption{The requirements on the SUIT architecture.}
        \label{tab:architecture-requirements}\\
        \toprule
        \begin{minipage}[b]{0.41\columnwidth}\raggedright\strut
        Requirement\strut
        \end{minipage} & \begin{minipage}[b]{0.53\columnwidth}\raggedright\strut
        Description\strut
        \end{minipage}\tabularnewline
        \midrule
        \endfirsthead
        \toprule
        \begin{minipage}[b]{0.41\columnwidth}\raggedright\strut
        Requirement\strut
        \end{minipage} & \begin{minipage}[b]{0.53\columnwidth}\raggedright\strut
        Description\strut
        \end{minipage}\tabularnewline
        \midrule
        \endhead

        \begin{minipage}[t]{0.41\columnwidth}\raggedright\strut
        Agnostic to how firmware images are distributed\strut
        \end{minipage} & \begin{minipage}[t]{0.53\columnwidth}\raggedright\strut
        The mechanism should not assume a particular technology is used to
        distributed manifests and images, but instead be able to be carried over
        different mediums.This means decisions about formats and distribution
        methods must not rely on features of a particular technology.\strut
        \end{minipage}\tabularnewline
        \begin{minipage}[t]{0.41\columnwidth}\raggedright\strut
        Friendly to broadcast delivery\strut
        \end{minipage} & \begin{minipage}[t]{0.53\columnwidth}\raggedright\strut
        The mechanism should be broadcast-friendly, meaning the mechanism can
        not be dependent on security on the transport layer or below. Also,
        devices receiving broadcast updates not meant for them should not
        incorrectly apply the update.\strut
        \end{minipage}\tabularnewline
        \begin{minipage}[t]{0.41\columnwidth}\raggedright\strut
        Use state-of-the-art security mechanisms\strut
        \end{minipage} & \begin{minipage}[t]{0.53\columnwidth}\raggedright\strut
        The SUIT specification assumes a PKI is in place. The PKI allows for
        trusted communication.\strut
        \end{minipage}\tabularnewline
        \begin{minipage}[t]{0.41\columnwidth}\raggedright\strut
        Rollback attacks must be prevented\strut
        \end{minipage} & \begin{minipage}[t]{0.53\columnwidth}\raggedright\strut
        The manifest should contain metadata such as monotonically increasing
        sequence numbers and best-before timestamps to avoid rollback
        attacks.\strut
        \end{minipage}\tabularnewline
        \begin{minipage}[t]{0.41\columnwidth}\raggedright\strut
        High reliability\strut
        \end{minipage} & \begin{minipage}[t]{0.53\columnwidth}\raggedright\strut
        The act of upgrading an image should not cause the device to break. This
        is an implementation requirement.\strut
        \end{minipage}\tabularnewline
        \begin{minipage}[t]{0.41\columnwidth}\raggedright\strut
        Operate with a small bootloader\strut
        \end{minipage} & \begin{minipage}[t]{0.53\columnwidth}\raggedright\strut
        The bootloader should be minimal. This is also an implementation
        requirement.\strut
        \end{minipage}\tabularnewline
        \begin{minipage}[t]{0.41\columnwidth}\raggedright\strut
        Small parser\strut
        \end{minipage} & \begin{minipage}[t]{0.53\columnwidth}\raggedright\strut
        It must be easy to parse the fields of the update manifest, as a large
        parser can get quite complex. Validation of the manifest will happen on
        the constrained devices, which further motivates a small parser and thus
        less complex manifests.\strut
        \end{minipage}\tabularnewline
        \begin{minipage}[t]{0.41\columnwidth}\raggedright\strut
        Minimal impact on existing firmware formats\strut
        \end{minipage} & \begin{minipage}[t]{0.53\columnwidth}\raggedright\strut
        The update mechanism itself must not make assumptions on the current
        format of firmware images, but be able to support different types of
        firmware image formats.\strut
        \end{minipage}\tabularnewline
        \begin{minipage}[t]{0.41\columnwidth}\raggedright\strut
        Robust permissions\strut
        \end{minipage} & \begin{minipage}[t]{0.53\columnwidth}\raggedright\strut
        Updates must be authorized before they are applied, and different
        configurations might have different requirements for authorization. The
        architecture should enable a flexible and robust permission model.\strut
        \end{minipage}\tabularnewline
        \begin{minipage}[t]{0.41\columnwidth}\raggedright\strut
        Operating modes\strut
        \end{minipage} & \begin{minipage}[t]{0.53\columnwidth}\raggedright\strut
        The draft presents three broad modes of updates: client-initiated
        updates, server-initiated updates, and hybrid updates, where hybrids are
        mechanisms that require interaction between the device and firmware
        provider before updating. This thesis will look into all three of these
        broad classes. Some classes may be preferred over others based on the
        technologies chosen in this thesis.\strut
        \end{minipage}\tabularnewline
        \bottomrule
\end{longtable}
\end{small}
        

The distribution of manifest and firmware images is also discussed, with two options being
possible. These are shown in Figure~\ref{fig:manifest-image-combined-distribution} and
Figure~\ref{fig:manifest-image-separate-distribution}. The first figure shows the manifest
and image distributed together to a firmware server. The device then receives the manifest
either via pulling or pushing and can subsequently download the image from the same
server. Alternatively, as shown in the second figure, the manifest itself can be directly
sent to the device without the need of a firmware server, while the firmware image is put
on the firmware server. After the device has received the lone manifest through some
method, the firmware can be downloaded from the firmware server. The SUIT architecture
does not enforce a specific method to be used when delivering the manifest and firmware,
but states that an update mechanism must support both types.

\begin{figure}[h]
    \caption{Distributing both manifest and image through a firmware server.}   
    \label{fig:manifest-image-combined-distribution}
    \includegraphics{images/together.pdf}
\end{figure}

\begin{figure}
    \caption{Distributing the manifest directly to the device and image through a firmware server.}
    \label{fig:manifest-image-separate-distribution}
    \includegraphics{images/separate.pdf}
\end{figure}

\subsection{Information Model}
\label{ssec:information-model}
The Internet-Draft for the SUIT information model presents the information needed in the
manifest to secure a firmware update mechanism \parencite{suit-information-model}. As
mentioned in the previous section, Internet-Drafts are works in progress and the thesis
references the information model draft from January 18, 2019. A manifest is needed for a
device to make a decision about whether or not to update itself, and whether or not the
image related to the manifest is valid and its integrity ensured. The draft also presents
threats and maps security requirements to them. Finally, it presents use cases and maps
usability requirements to them in order to motivate the presence of each manifest element.
Note that the information model does not discuss threats outside of transporting the
updates, such as physical attacks.

The proposed mandatory and recommended manifest elements and their brief motivations can
be seen in Table~\ref{tab:manifest-elements}. For the optional elements and more detailed
motivations, use cases, and requirements refer to \parencite{suit-information-model}.

\begin{small}
\begin{longtable}[]{@{}lll@{}}
    \caption[The proposed mandatory and recommended manifest elements of the SUIT information model.]
        {The proposed mandatory and recommended manifest elements of the SUIT information model. Adapted from \parencite{suit-information-model}.}
    \label{tab:manifest-elements}\\
    \toprule
    \begin{minipage}[b]{0.23\columnwidth}\raggedright\strut
    Manifest Element\strut
    \end{minipage} & \begin{minipage}[b]{0.26\columnwidth}\raggedright\strut
    Status\strut
    \end{minipage} & \begin{minipage}[b]{0.42\columnwidth}\raggedright\strut
    Explanation\strut
    \end{minipage}\tabularnewline
    \midrule
    \endfirsthead
    \toprule
    \begin{minipage}[b]{0.23\columnwidth}\raggedright\strut
    Manifest Element\strut
    \end{minipage} & \begin{minipage}[b]{0.26\columnwidth}\raggedright\strut
    Status\strut
    \end{minipage} & \begin{minipage}[b]{0.42\columnwidth}\raggedright\strut
    Explanation\strut
    \end{minipage}\tabularnewline
    \midrule
    \endhead

    \begin{minipage}[t]{0.23\columnwidth}\raggedright\strut
    Version identifier\strut
    \end{minipage} & \begin{minipage}[t]{0.26\columnwidth}\raggedright\strut
    Mandatory\strut
    \end{minipage} & \begin{minipage}[t]{0.42\columnwidth}\raggedright\strut
    Describes the iteration of the manifest format\strut
    \end{minipage}\tabularnewline
    \begin{minipage}[t]{0.23\columnwidth}\raggedright\strut
    Monotonic Sequence Number\strut
    \end{minipage} & \begin{minipage}[t]{0.26\columnwidth}\raggedright\strut
    Mandatory\strut
    \end{minipage} & \begin{minipage}[t]{0.42\columnwidth}\raggedright\strut
    Prevents rollbacks to older images\strut
    \end{minipage}\tabularnewline
    \begin{minipage}[t]{0.23\columnwidth}\raggedright\strut
    Payload Format\strut
    \end{minipage} & \begin{minipage}[t]{0.26\columnwidth}\raggedright\strut
    Mandatory\strut
    \end{minipage} & \begin{minipage}[t]{0.42\columnwidth}\raggedright\strut
    Describes the format of the payload\strut
    \end{minipage}\tabularnewline
    \begin{minipage}[t]{0.23\columnwidth}\raggedright\strut
    Storage Location\strut
    \end{minipage} & \begin{minipage}[t]{0.26\columnwidth}\raggedright\strut
    Mandatory\strut
    \end{minipage} & \begin{minipage}[t]{0.42\columnwidth}\raggedright\strut
    Tells the device which component is being updated, can be used to
    establish physical location of update\strut
    \end{minipage}\tabularnewline
    \begin{minipage}[t]{0.23\columnwidth}\raggedright\strut
    Payload Digest\strut
    \end{minipage} & \begin{minipage}[t]{0.26\columnwidth}\raggedright\strut
    Mandatory\strut
    \end{minipage} & \begin{minipage}[t]{0.42\columnwidth}\raggedright\strut
    The digest of the payload to ensure authenticity. Must be possible to
    specify more than one payload digest.\strut
    \end{minipage}\tabularnewline
    \begin{minipage}[t]{0.23\columnwidth}\raggedright\strut
    Size\strut
    \end{minipage} & \begin{minipage}[t]{0.26\columnwidth}\raggedright\strut
    Mandatory\strut
    \end{minipage} & \begin{minipage}[t]{0.42\columnwidth}\raggedright\strut
    The size of the payload in bytes\strut
    \end{minipage}\tabularnewline
    \begin{minipage}[t]{0.23\columnwidth}\raggedright\strut
    Signature\strut
    \end{minipage} & \begin{minipage}[t]{0.26\columnwidth}\raggedright\strut
    Mandatory\strut
    \end{minipage} & \begin{minipage}[t]{0.42\columnwidth}\raggedright\strut
    The manifest is to be wrapped in an authentication container (not a
    manifest element itself)\strut
    \end{minipage}\tabularnewline
    \begin{minipage}[t]{0.23\columnwidth}\raggedright\strut
    Dependencies\strut
    \end{minipage} & \begin{minipage}[t]{0.26\columnwidth}\raggedright\strut
    Mandatory\strut
    \end{minipage} & \begin{minipage}[t]{0.42\columnwidth}\raggedright\strut
    A list of digest/URI pairs linking manifests that are needed to form a
    complete update\strut
    \end{minipage}\tabularnewline
    \begin{minipage}[t]{0.23\columnwidth}\raggedright\strut
    Precursor Image Digest Condition\strut
    \end{minipage} & \begin{minipage}[t]{0.26\columnwidth}\raggedright\strut
    Mandatory (for differential updates)\strut
    \end{minipage} & \begin{minipage}[t]{0.42\columnwidth}\raggedright\strut
    If a precursor image is required, this digest condition is needed\strut
    \end{minipage}\tabularnewline
    \begin{minipage}[t]{0.23\columnwidth}\raggedright\strut
    Content Key Distribution Method\strut
    \end{minipage} & \begin{minipage}[t]{0.26\columnwidth}\raggedright\strut
    Mandatory (for encrypted payloads)\strut
    \end{minipage} & \begin{minipage}[t]{0.42\columnwidth}\raggedright\strut
    Tells how keys for encryption/decryption are distributed\strut
    \end{minipage}\tabularnewline
    \begin{minipage}[t]{0.23\columnwidth}\raggedright\strut
    Vendor ID Condition\strut
    \end{minipage} & \begin{minipage}[t]{0.26\columnwidth}\raggedright\strut
    Recommended\strut
    \end{minipage} & \begin{minipage}[t]{0.42\columnwidth}\raggedright\strut
    Helps distinguish products from different vendors\strut
    \end{minipage}\tabularnewline
    \begin{minipage}[t]{0.23\columnwidth}\raggedright\strut
    Class ID Condition\strut
    \end{minipage} & \begin{minipage}[t]{0.26\columnwidth}\raggedright\strut
    Recommended\strut
    \end{minipage} & \begin{minipage}[t]{0.42\columnwidth}\raggedright\strut
    Helps distinguish incompatible devices in a vendors infrastructure\strut
    \end{minipage}\tabularnewline
    \bottomrule
\end{longtable}
\end{small}

As the SUIT architecture aims to be a standardized solution it must account for different
use cases and different combinations of use cases. As a result, many of the mandatory and
recommended elements are there to enable certain use cases that might not always be
relevant. This means certain information must be prepared for in advance even if it is not
going to be used in all cases. There is a trade off with flexibility and size, and as the
devices of interest are lightweight, it is of interest to reduce the size of the manifest
as much as possible, without limiting use cases. With these two considerations in mind, a
manifest for the architecture proposed in this thesis must be designed to facilitate as
many different use cases as possible while keeping sizes to a minimum.

To summarize, the SUIT information model proposes the use of a signed manifest that is
distributed through some method to each device in need of an update. The device then
parses the manifest in order to determine if the update is trusted, suitable, and
up-to-date, with many other optional elements such as special processing steps, and new
URIs to fetch the images. The model does not make assumptions about technology which is
one of the reasons there are optional elements. Not all of them are applicable to all
solutions. Nevertheless, the architecture and information model together provides a solid
base on which to design a secure update mechanism for IoT.

\section{Contiki-NG}
\label{sec:contiki-ng}
Contiki-NG is an open-source operating system for resource constrained IoT devices based
on the Contiki operating system \parencite{contiki-ng-github, contiki-github}. Contiki-NG
features an IPv6 network stack designed for unreliable, low-power IoT networks. There are
many protocols implemented in the stack. This thesis will look at UDP and CoAP secured by
DTLS. Beneath IPv6 Contiki-NG supports IEEE 802.15.4 with Time Slotted Channel Hopping
\parencite{ieee-802.15.4}.

The CoAP implementation in Contiki-NG is based on Erbium by Mattias Kovatsch and supports
both unsecured (CoAP) and secured (CoAPs) communication \parencite{low-power-coap}. CoAPs
uses a DTLS implementation called TinyDTLS which handles encryption and decryption of
messages \parencite{tinydtls-github}.

Contiki-NG has a process abstraction built on lightweight protothreads
\parencite{protothreads}. Protothreads can be seen as a combination of threads and
event-driven programming. Threads provide a way of running sequential programs
concurrently, enabling the use of flow-control mechanisms. This, however, requires a
thread-local stack which is too resource heavy on small IoT devices. Event-driven
programming, on the other hand, does not require a thread-local stack, but programs are
limited to computation as callbacks on event triggers, meaning sequential programs are
more difficult to program. Protothreads combine these paradigms in a lightweight way,
keeping the yield semantics of threads and stacklessness of event-driven programming. All
protothreads in a system are run on the same stack, meaning each protothread has a very
low memory overhead, and by providing a conditional blocking wait statement, protothreads
can execute cooperatively.

\begin{figure}
    \caption[The state machine of Contiki threads.]
        {The state machine of Contiki threads. Adapted from \parencite{contiki-multithreading}.}
    \label{fig:state-machine}
    \begin{tikzpicture}[->,>=stealth',shorten >=1pt,auto,node distance=5cm, semithick]
        \node[initial,state] (A) {Ready};
        \node[state] (C) [below right of=A] {Exited};
        \node[state] (B) [above right of=C] {Running};
    
        \path (A) edge node {stop} (C)
                  edge[bend left=15] node {exec} (B)
              (B) edge[bend left=15] node {yield} (A)
                  edge node {stop} (C);
    \end{tikzpicture}
\end{figure}

Contiki-NGs execution model is event-based, meaning processes often yield execution until
they are informed a certain event has taken place, upon which they can act.
Figure~\ref{fig:state-machine} shows the states of the processes and their transitions.
User-space processes are run in a cooperative manner while kernel-space processes can
preempt user-space processes. Examples of events are timers expiring, a process being
polled, or a network packet arriving.

Contiki-NG provides two memory allocators in addition to using static memory. They are
called \texttt{MEMB} and \texttt{HeapMem} and are semi-dynamic and dynamic, respectively.
\texttt{MEMB} is a semi-dynamic allcator that allocates pools of static memory as arrays
of constant-sized objects. After a memory pool has been declared, it is initialized, after
which objects can be allocated memory from the pool. All objects allocated through the
same pool have the same size. \texttt{HeapMem} solves the issue of dynamically allocating
objects of varying sizes during runtime in Contiki-NG. It can be used on a variety of
hardware platforms, something a standard C \texttt{malloc} implementation could struggle
with. With \texttt{HeapMem}, memory can be reallocated and deallocated as if using a normal
\texttt{malloc}.

\section{Summary}
\label{sec:2-summary}
This chapter introduced and motivated the use of network protocols common in IoT networks.
The SUIT specification for providing updates, which is agnostic to any specific
technology, was also introduced. This standard can be expanded and profiled in order to
propose a possible solution. This thesis develops a prototype of such a solution using the
Contiki-NG operating system. The next chapter expands upon the works of SUIT, discussing
topics such as roles, authorization, and communication of updates, applying a life cycle
view to IoT updates, and defining profiles for implementing an update architecture.

\end{document}