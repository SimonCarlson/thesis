\documentclass[0-thesis.tex]{subfiles}

\begin{document}
The previous chapter introduced the update architecture proposed by the thesis and the
design of a prototype aimed to evaluate it. This chapter will explain how the evaluation
was carried out, its results, and discussion about the results.

\section{Quantitative Evaluation of Update Architecture}
\label{sec:quant-evaluation}

\section{Qualitative Evaluation of Update Architecture}
\label{sec:qual-evaluation}

\subsection{Architecture}
\label{ssec:arch-evaluation}
\begin{description}
    \item[Agnostic to how firmware images are distributed:]
        The architecture does not make assumptions on underlying transport. The thesis
        gives examples of two different profiles using different means of transport.
    \item[Friendly to broadcast delivery:]
        Nothing in the architecture prevents broadcasting, however choice of security can
        limit it (for instance, DTLS).
    \item[Use state-of-the-art security mechanisms:]
        The architecture is based on asymmetric cryptography using strong algorithms
        (preferably ECC) and possibly tokens for fine-grained authorization. 
    \item[Rollback attacks must be prevented:]
        Prevented via sequence numbers in the manifest. For each targeted class of
        devices, the sequence number of updates must be monotonically increasing (also
        version numbers?)
    \item[High reliability:]
        This is an implementation specific requirement, however the storage element of the
        manifest aids in achieving safe storage of a new image. After a successful update,
        devices are to re-register at servers. No acknowledgment means the server knows
        the update still must be applied, thus an interrupted update can be redistributed.
    \item[Operate with a small bootloader:]
        The thesis suggests to store an unencrypted image alongside its digest for the
        bootloader to be minimal, only needing support for SHA256. All information about
        whether or not to perform the update is encoded in conditions in the manifest, and
        can be stored with minimal memory usage.  
    \item[Small parsers:]
        The manifest format used in the thesis is minimal while still complying with SUIT,
        easily parsed, and extensible for extra functionality.
    \item[Minimal impact on existing firmware formats:]
        The architecture makes no assumptions about firmware formats.
    \item[Robust permissions:]
        The architecture proposes whitelists of operators and servers that device use to
        permit traffic. Authorization tokens are suggested as a way of achieving more
        finely grained level of authorization, for instance for differential updates.
        Different deployments can or different devices in the same deployment can have
        different authorization configurations.
    \item[Operating modes:]
        The architecture supports the device initiated pull model, as well as the operator
        initiated push model. The update server acts as a mediator between device and
        operator as well as a repository for images and device profiles, letting the
        operator query the server for device statuses.
\end{description}
  
\subsection{Information Model}
\label{ssec:information-evaluation}
  
\end{document}