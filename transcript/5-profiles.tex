\documentclass[0-thesis.tex]{subfiles}

\begin{document}
\label{chap:profiles}
% TODO: Write short introduction
This chapter will 

\section{DTLS/CoAP profile}
\label{sec:dtls-coap-profile}
When implementing the architecture, one choice of protocols could be using DTLS and CoAP
for constrained communication between operators or servers and devices. There are existing
profiles for TLS and DTLS in IoT contexts, see \parencite{rfc7925}. These profiles define
credential types, error handling, compression and so on for TLS and DTLS. The DTLS/CoAP
profile of the proposed architecture aims to stay as close as possible to the technologies
presented in the DTLS IoT profiles while still complying with the requirements of SUIT.
The DTLS/CoAP profile of the proposed architecture needs to cover four things:

\begin{itemize}
    \item Key algorithms for encrypted communication
    \item Hashing algorithms for image digests
    \item Encoding of manifest and image
    \item Encryption of manifest and image
\end{itemize}

As the architecture will use both pre-shared keys for enrollment and then certificates,
credential types for both of these modes must be specified. The CoAP specification
specifies the ciphersuite TLS\_PSK\_WITH\_AES\_128\_CCM\_8{} to be mandatory to implement
for CoAP \parencite{rfc7252}. For certificates, the (D)TLS profiles in \parencite{rfc7925}
recommends the use of TLS\_ECDHE\_ECDSA\_WITH\_AES\_128\_CCM\_8{} using the secp256r1
curve. TLS\_PSK\_WITH\_AES\_128\_CCM\_8{} uses SHA-256 for hashing and
TLS\_ECDHE\_ECDSA\_WITH\_AES\_128\_CCM\_8{} should support SHA-256 as well
\parencite{rfc7251}. 

These ciphersuites can be used for pre-shared keys and certificate keys in the update
architecture, meaning devices that already support CoAP and DTLS do not have to implement
additional cryptography libraries. Entities communicating with each other outside the
constrained network such as operators and servers can use HTTP over TLS and thus use the
same ciphersuites for all kinds of communication. Since these ciphersuites enforce the use
of SHA-256 for hashing, SHA-256 can be used to calculate image digests.

For encoding, \gls{cbor} is an efficient binary encoding \parencite{rfc7049}. CBOR aims to
be a extensible data format providing very small code sizes. It supports simple values as
well as arrays and maps meaning it is easy to map to and from JSON while being more
compact than JSON. The use of CBOR also enables the use of \gls{cose}. COSE provides
encryption for CBOR encoded objects, which in the architecture will be the manifest and
image.

% TODO: Asymmetric/symmetric keys for manifest/image?

For enrollment, EST-coaps is a suitable protocol that can be used to (re)enroll and
transport certificates or keys in case of server-side key generation
\parencite{est-coaps}. EST has already been profiled to fit with DTLS. For authorization
tokens, ACE is a fitting protocol. Section 6.4 of \parencite{ace} states that "There may
be use cases were different profiles of this framework are combined.  For example, an
MQTT-TLS profile is used between the client and the RS in combination with a CoAP-DTLS
profile for interactions between the client and the AS.  Ideally, profiles should be
designed in a way that the security of system should not depend on the specific security
mechanisms used in individual protocol interactions." This fits with the proposed
architecture as all entities involved will need to authorize but will use different
protocols to communicate. The security of the architecture is however not bound to
characteristics of any of these protocols, ensuring the architecture is not easily
attacked.

% TODO: Broadcasting woes

\end{document}