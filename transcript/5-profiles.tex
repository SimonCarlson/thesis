\documentclass[0-thesis.tex]{subfiles}

\begin{document}
\label{chap:profiles}
% TODO: Restructure chapter again. Point to existing profiles for DTLS, EST-coaps, ACE.
% Specify signing and encoding of payloads?

% TODO: Write short introduction
When implementing the architecture, one choice of protocols is using DTLS and CoAP for
constrained communication between operators or servers and devices. There are existing
profiles for TLS and DTLS in IoT contexts, see \parencite{rfc7925}. These profiles define
credential types, error handling, compression and so on for TLS and DTLS. The DTLS/CoAP
profile of the proposed architecture aims to stay as close as possible to the technologies
presented in the DTLS IoT profiles while still complying with the requirements of SUIT.
The DTLS/CoAP profile of the proposed architecture needs to cover three things:

\begin{itemize}
    \item Key algorithms for encrypted communication
    \item Hashing algorithms for image digests
    \item Encoding and encrypting manifest and image
\end{itemize}


\section{Ciphersuites and Hash Algorithms}
\label{sec:ciphersuites-hash}
As the architecture will use both pre-shared keys for enrollment and then certificates,
credential types for both of these modes must be specified. The (D)TLS profiles specifies
ciphersuite TLS\_PSK\_WITH\_AES\_128\_CCM\_8{} pre-shared keys and
TLS\_ECDHE\_ECDSA\_WITH\_AES\_128\_CCM\_8{} for certificates using the secp256r1 elliptic
curve. TLS\_PSK\_WITH\_AES\_128\_CCM\_8{} uses SHA-256 for hashing and
TLS\_ECDHE\_ECDSA\_WITH\_AES\_128\_CCM\_8{} should support SHA-256 as well
\parencite{rfc7251}. 

These ciphersuites can be used for pre-shared keys and certificate keys for securing
communication in the update architecture, meaning devices that already support CoAP and
DTLS do not have to implement additional cryptography libraries. Entities communicating
with each other outside the constrained network such as operators and servers can use HTTP
over TLS and thus use the same ciphersuites for all kinds of communication. Since these
ciphersuites enforce the use of SHA-256 for hashing, SHA-256 can be used to calculate
image digests.

\section{Encoding and Encryption of Application Data}
\label{sec:encoding-encryption}
For encoding, \gls{cbor} is an efficient binary encoding \parencite{rfc7049}. CBOR aims to
be a extensible data format providing very small code sizes. It supports simple values as
well as arrays and maps meaning it is easy to map to and from JSON while being more
compact than JSON. The use of CBOR also enables the use of \gls{cose}. COSE provides
encryption for CBOR encoded objects, which in the architecture will be the manifest and
image. The COSE standard describes how to encrypt or sign payloads using COSE objects and
also handles message authentication codes.

The architecture does not impose restrictions towards broadcasting updates, but as this
particular profile is based on the security granted by DTLS it does not support
broadcasting. DTLS only provides transport layer security which is not compatible with
broadcasting messages. In order to support broadcasting, OSCORE can be used instead.

% TODO: Asymmetric/symmetric keys for manifest/image?

\gls{oscore} provides end-to-end protection for CoAP messages using COSE. By providing
end-to-end encryption CoAP message security is not terminated at a proxy or update server
as with DTLS. This means a compromised proxy or update server cannot be leveraged to
eavesdrop or manipulate data. The difference between using OSCORE and using COSE objects
in normal CoAP is that OSCORE protects the entire CoAP message end-to-end while COSE and
CoAP protects the payloads through COSE signing but not the CoAP message. Without using
OSCORE, messages are only protected during transport.

\section{Enrollment and Authorization}
\label{sec:enrollment-authorization}
For enrollment, EST-coaps is a suitable protocol that can be used to (re)enroll and
transport certificates or keys in case of server-side key generation
\parencite{est-coaps}. EST has been profiled to fit with DTLS and also makes use of
TLS\_ECDHE\_ECDSA\_WITH\_AES\_128\_CCM\_8{} with the same elliptic curve. 

For authorization tokens, ACE is a fitting protocol. Section 6.4 of \parencite{ace} states
that "There may be use cases were different profiles of this framework are combined.  For
example, an MQTT-TLS profile is used between the client and the RS in combination with a
CoAP-DTLS profile for interactions between the client and the AS.  Ideally, profiles
should be designed in a way that the security of system should not depend on the specific
security mechanisms used in individual protocol interactions." This fits with the proposed
architecture as all entities involved will need to authorize but might use different
protocols to communicate. The security of the architecture is however not bound to
characteristics of any of these protocols, ensuring the architecture is not easily
compromised. ACE has also been profiled to fit with DTLS \parencite{ace-dtls-profile}.

At the time of writing, there are two Internet-Drafts aiming to specify profiles for ACE
and EST using OSCORE \parencite{ace-oscore, est-oscore}.

\section{Summary}
To summarize, DTLS with CoAP can be used for communication in a constrained network.
DTLS-based security terminates at a proxy or server and is thus not end-to-end, which also
means broadcasting is not supported. Instead OSCORE can be used which provides end-to-end
security. For pre-shared keys the ciphersuite TLS\_PSK\_WITH\_AES\_128\_CCM\_8{} is
proposed and for certificates TLS\_ECDHE\_ECDSA\_WITH\_AES\_128\_CCM\_8{}. For calculating
digests SHA-256 is proposed. Payloads can be encoded and encrypted using COSE no matter if
DTLS or OSCORE is used. For enrollment and authorization, EST and ACE are proposed, with
standards for CoAP/OSCORE profiles for EST and ACE  being developed at the time of
writing.
\end{document}