\documentclass[0-thesis.tex]{subfiles}

\begin{document}
\label{chap:profiles}
% TODO: Write short introduction
This chapter will 

\section{DTLS/CoAP profile}
\label{sec:dtls-coap-profile}
When implementing the architecture, one choice of protocols could be using DTLS and CoAP
for constrained communication. There are existing profiles for TLS and DTLS in IoT
contexts, see \parencite{rfc7925}. These profiles define credential tyes, error
handling, compression and so on for TLS and DTLS. The DTLS/CoAP profile of the proposed
architecture aims to stay as close as possible to the technologies presented in the DTLS
IoT profiles while still complying with the requirements of SUIT. The DTLS/CoAP profile of
the proposed architecture needs to cover four things:

\begin{itemize}
    \item Key algorithms for encrypted communication
    \item Hashing algorithms for image digests
    \item Encoding of manifest and image
    \item Encryption of manifest and image
\end{itemize}

As the architecture will use both pre-shared keys for enrollment and then certificates,
credential types for both of these modes must be specified. The CoAP specification
specifices the ciphersuite TLS\_PSK\_WITH\_AES\_128\_CCM\_8{} to be mandatory to implement
for CoAP \parencite{rfc7252}. For certificates, the (D)TLS profiles in \parencite{rfc7925}
recommends the use of TLS\_ECDHE\_ECDSA\_WITH\_AES\_128\_CCM\_8{} using the secp256r1
curve. TLS\_PSK\_WITH\_AES\_128\_CCM\_8{} uses SHA-256 for hashing and
TLS\_ECDHE\_ECDSA\_WITH\_AES\_128\_CCM\_8{} should support SHA-256 as well \parencite{rfc7251}.
The same algorithms can be used for pre-shared keys and certificate keys in the update
architecture, meaning devices that already support CoAP and DTLS do not have to
reimplement cryptography libraries. Since these ciphersuites enforce the use of SHA-256
for hashing, SHA-256 can be used to calculated image digests.

For encoding, \gls{cbor} is an efficient binary encoding \parencite{rfc7049}. CBOR aims to
be a extensible data format providing very small code sizes. It supports simple values as
well as arrays and maps meaning it is easy to map to and from JSON while being more
compact than JSON. The use of CBOR also enables the use of \gls{cose}. COSE provides
encryption for CBOR encoded objects, which in the architecture will be the manifest and
image.



% TODO: Asymmetric/symmetric keys for manifest/image?

% TODO: 
% EST-coaps for enrollment
% ACE for authorization
% HTTP/CoAP for application
% TLS/DTLS for transport
% Broadcasting woes

\end{document}