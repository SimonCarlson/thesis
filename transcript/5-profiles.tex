\documentclass[0-thesis.tex]{subfiles}

\begin{document}
\label{chap:profiles}
When implementing the architecture, one choice of protocols is using DTLS and CoAP for
constrained communication to and from devices. For enrollment and authorization, EST and
ACE can be used with their respective suitable profiles. However, as DTLS provides
security at the transport layer it cannot be used for broadcasting, \gls{oscore} is an
option better suited for broadcasting. This chapter will discuss choice of image digest
algorithm, vendor and class ID generation, payload encoding and encryption, and
considerations when choosing between DTLS/CoAP and OSCORE for an update architecture
profile.

DTLS is already profiled for use in IoT contexts and EST and ACE profiles are works in
progress \parencite{rfc7925, est-coaps, ace-dtls-profile}. The update architecture needs
little extra profiling in addition to these protocols, namely image digest algorithm,
payload encoding, and payload encryption. The security of the architecture is not bound to
characteristics of any of these protocols and others can be used instead, this is just one
possible profile.

The DTLS profile suggests ciphersuites for the use of pre-shared keys and certificates and
EST-coaps has been profiled to use the same ciphersuites for certificates. These suites
implement the \textbf{SHA-256} hash function which can be used to calculate the digest of
an image. Since the digest is just supposed to verify the integrity of the image, SHA-256
is an adequate choice and does not require implementing new hash algorithms for devices
already running DTLS.

% TODO: UUID generation?

\section{Encoding and Encryption of Application Data}
\label{sec:encoding-encryption}
For encoding, \gls{cbor} is an efficient binary encoding \parencite{rfc7049}. CBOR aims to
be an extensible data format providing very small code sizes. It supports simple values as
well as arrays and maps meaning it is easy to map to and from JSON while being more
compact than JSON. 

The use of CBOR also enables the use of \gls{cose}. COSE provides encryption and signing
for CBOR encoded objects, which in the architecture will be the manifest and image
\parencite{rfc8152}. As DTLS secures the channel, the manifest and image can be signed
using COSE objects to ensure their integrity during transport. 

\section{DTLS/CoAP or OSCORE?}
\label{sec:dtls-coap-or-oscore}
The main difference between using CoAP over DTLS and OSCORE is on which level the
communication is secured. DTLS provides transport layer security setting up a secure
channel between each hop in the protected network. This means payloads are encrypted but
the secure channel ends at for instance a proxy or update server. DTLS does not provide
end-to-end security and as the security provided is on a transport level it is not
compatible with broadcasting. The architecture itself does not impose restrictions towards
broadcasting updates, but a profile based on the security granted by DTLS will not support
it. In order to support broadcasting, OSCORE can be used instead.

OSCORE provides end-to-end protection for CoAP messages using COSE \parencite{oscore}. By
providing end-to-end encryption CoAP message security is not terminated at a proxy or
update server as with DTLS. This means a compromised proxy or update server cannot be
leveraged to eavesdrop or manipulate data. The difference between using OSCORE and using
COSE objects in CoAP with DTLS is that OSCORE protects the entire CoAP message end-to-end
while COSE and CoAP protects the payloads through COSE signing but not the CoAP message
itself, it relies on a secure channel established by DTLS.

OSCORE can run directly on top of UDP and supports broadcasting as well as unicast. In the
case of broadcasting, a security context must be defined \parencite{oscore-group}.
Defining a security context is out of scope for the architecture as it is for a specific
piece of technology. If a security context is in place, unicast OSCORE is still possible.
At the time of writing, there are two Internet-Drafts aiming to specify profiles for ACE
and EST using OSCORE, meaning these protocols can be used for enrollment and authorization
with OSCORE \parencite{ace-oscore, est-oscore}.

Broadcasting is of interest in the update architecture as communication between update
servers and devices can be one-to-one, many-to-one, one-to-many, or many-to-many. It is
dependant on the topology of the network and how update servers are mapped to devices and
thus logically dividing the network. In order to efficiently update many devices of the
same class or from the same vendor, broadcasting can be employed. The information in the
manifest makes sure that devices receiving broadcasted updates not intended for them does
not erroneously install the update.

While being better suited for broadcasting and providing end-to-end security, OSCORE is
not yet standardised. The specification (\parencite{oscore}) is a work in progress as is
many of the profiles mentioned in this chapter. DTLS and its IoT profiles are standardised
and Contiki-NG features a fully working DTLS implementation. If implementing the
architecture using OSCORE, it is worth noting the incomplete status of the standard.

\section{Summary}
\label{sec:profile-summary}
This chapter suggested the use of DTLS, CoAP, EST-coaps, and ACE DTLS profiles to realize
the update architecture. SHA-256 is the algorithm of choice for calculating image digests,
and payloads can be encoded and signed using CBOR and COSE respectively. DTLS does not
support broadcasting however which might be of importance, for this purpose OSCORE can be
used instead. OSCORE does not rely on a secure channel established by DTLS but instead
provides end-to-end security for CoAP messages using COSE. OSCORE, unlike DTLS, is not yet
standardised and the author is not aware of any OSCORE implementation in Contiki-NG. 

\end{document}