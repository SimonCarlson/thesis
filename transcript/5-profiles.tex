\documentclass[0-thesis.tex]{subfiles}

\begin{document}
\label{chap:profiles}
When implementing the architecture, one choice of protocols is using DTLS and CoAP for
constrained communication between operators or servers and devices. For enrollment and
authorization, EST-coaps and ACE can be used. As DTLS provides security at the transport
layer it cannot be used for broadcasting. \gls{oscore} is an option better suited for
broadcasting as it provides end-to-end security of messages and is not bound to the secure
channel of DTLS. This chapter will discuss choice of hash algorithm, payload encoding and
encryption, and considerations when choosing between DTLS/CoAP and OSCORE for a update
architecture profile.

DTLS is already profiled for use in IoT contexts and EST and ACE profiles are work in
progress \parencite{rfc7925, est-coaps, ace-dtls-profile}. The update architecture needs
little extra profiling in addition to these protocols, namely image digest algorithms,
payload encoding, and payload encryption. The security of the architecture is not bound to
characteristics of any of these protocols and others can be used instead, this is just one
possible profile.

The DTLS profile suggests ciphersuites for the use of pre-shared keys and certificates and
EST-coaps has been profiled to use the same ciphersuites for certificates. These suites
implement the \textbf{SHA-256} hash function which can be used to calculate the digest of an image.
Since the digest is just supposed to verify the integrity of the image, SHA-256 is a fine
choice and does not require implementing new hash algorithms for devices already running
DTLS.

\section{Encoding and Encryption of Application Data}
\label{sec:encoding-encryption}
For encoding, \gls{cbor} is an efficient binary encoding \parencite{rfc7049}. CBOR aims to
be a extensible data format providing very small code sizes. It supports simple values as
well as arrays and maps meaning it is easy to map to and from JSON while being more
compact than JSON. 

The use of CBOR also enables the use of \gls{cose}. COSE provides encryption and signing
for CBOR encoded objects, which in the architecture will be the manifest and image. As
DTLS secures the channel, the manifest and image can be signed using COSE objects to
ensure their integrity during transport. 

\section{Broadcasting}
\label{sec:broadcasting}
The architecture itself does not impose restrictions towards broadcasting updates, but a
profile based on the security granted by DTLS will not support broadcasting. DTLS only
provides transport layer security which is not compatible with broadcasting messages. In
order to support broadcasting, OSCORE can be used instead.

Broadcasting is of interest in the update architecture as communication between update
servers and devices can be one-to-one, many-to-one, one-to-many, or many-to-many. It is
dependant on the topology of the network and how update servers are mapped to devices and
thus logically dividing the network. In order to efficiently update many devices of the
same class or from the same vendor, broadcasting can be employed. The information in the
manifest makes sure that devices receiving broadcasted updates not intended for them does
not erroneously install the update.

OSCORE provides end-to-end protection for CoAP messages using COSE. By providing
end-to-end encryption CoAP message security is not terminated at a proxy or update server
as with DTLS. This means a compromised proxy or update server cannot be leveraged to
eavesdrop or manipulate data. The difference between using OSCORE and using COSE objects
in CoAP with DTLS is that OSCORE protects the entire CoAP message end-to-end while COSE
and CoAP protects the payloads through COSE signing but not the CoAP message itself, it
relies on a secure channel established by DTLS.

OSCORE can run directly on top of UDP and supports broadcasting as well as unicast. In the
case of broadcasting, a security context must be defined \parencite{oscore-group}.
Defining a security context is out of scope for the architecture as it is for a specific
piece of technology. If a security context is in place, unicast OSCORE is still possible.
At the time of writing, there are two Internet-Drafts aiming to specify profiles for ACE
and EST using OSCORE \parencite{ace-oscore, est-oscore}.

\section{Summary}

\end{document}