\documentclass[0-thesis.tex]{subfiles}

\begin{document}
This chapter presents an update architecture based on the SUIT standard. The architecture
encompasses device management, end-to-end security for manifests and images,
transportation of manifests and images, and the life cycle of a device. Section hm hm does
ha ha

\section{Device Life Cycle}
\label{sec:device-lifecycle}
A network of IoT devices can consist of many different kinds of devices. They may contain
different components, perform different tasks, and thus have different means of
communication. This will impact how they enroll in a secure network and how they
communicate with a server, the server will need to know what protocols and security
measure a specific device handles. It is helpful to think about the life cycle of devices:
how they enroll in a secure network, how they securely communicate with a server, and how
they can be maintained for all those years they are intended to work.

The life cycle of a device can be summarized as in Figure \ref{fig:lifecycle}. The figure
shows the different stages of a device from being manufactured to ending its service.
There are also annotations showing what needs to be done in each stage. These stages are
discussed in further detail in this section.

\begin{figure}
    \caption{The life cycle of a device.}
    \label{fig:lifecycle}
    \begin{tikzpicture}[>=stealth',
        box/.style={rectangle,draw,minimum width=3cm,minimum height=2cm}]
        \node[box] (new) {Factory new};
        \node[box, below right=1.5cm of new] (enroll) {Enrollment stage};
        \node[box, below left=1.5cm of enroll] (maintain) {Maintenance};
        \node[box, below left=1.5cm of new] (end) {End of life};

        \node[align=left, above right=0.6cm of new] (new_notes) {Ship with:\\
            Pre-shared key\\
            Vendor and class IDs\\
            Server registration endpoint\\
            CA certificate};
        \node[align=left, xshift=-2.5cm, yshift=-1cm, below right=0.4cm of enroll] (enroll_notes) 
            {Register at server\\
            Obtain certificate};
        \node[align=left, below left=0.7cm of maintain] (maintain_notes) {Push/pull updates\\
            Update server profile};

        \path[->,>=stealth]
        (new) edge [bend left=35] (enroll)
        (enroll) edge [bend left=35] (maintain)
        (maintain) edge [bend left=35] (end);
        
        \def\myshift#1{\raisebox{-2.5ex}}
        \path[decoration={text along path,text align=center,
        text={|\myshift|Please recycle}, raise=0.7cm}]
        (end) edge[decorate, bend left=35] node {} (new);

        \path[->,>=stealth,dashed]
        (end) edge [bend left=35] (new);

        \path[-,>=stealth]
        (new) edge (new_notes)
        (enroll) edge (enroll_notes)
        (maintain) edge (maintain_notes);
    \end{tikzpicture}
\end{figure}

A factory new device that is to be installed in an IoT network needs some information in
order to enroll and register at a server. As a basis for secure communication, a
pre-shared key is needed. This key can either be used for symmetric encryption or for
enrolling in an asymmetric encryption infrastructure. Both modes can be used and the
architecture does not care about which is used as long as end-to-end security of manifests
and images can be achieved. Symmetric encryption however will require a symmetric key for
each device in the network, and as networks can grow very large it will become
increasingly difficult to handle keys. Asymmetric encryption solves this, but instead
needs a trusted certificate authority to enroll devices. With the availability of
EST-coaps and therefore certificates in an IoT context, the thesis will discuss the
architecture from an asymmetric cryptography viewpoint, but keep in mind that purely
symmetric encryption would work as well.

With a pre-shared key, devices can enter the enrollment stage of the life cycle and
securely enroll and obtain a certificate. In order to trust this certificate, they also
need to be shipped with at least one certificate of the certificate authority, or of some
authority further down the chain of trust. If a device is not equipped with a certificate
it cannot trust the authority when enrolling and thus not be sure it has obtained a valid
certificate or not.

After a device has enrolled, it needs to register at a server. This is because IoT
networks can be heterogeneous, containing different kinds of devices communicating in
different ways. In order for a server to know how to reach a device, the device must tell
the server if its capabilities. As discussed in Section \ref{ssec:information-model}, the
manifest should contain a vendor and class ID so a device can check if an update is
applicable, this means a device must be aware of its own vendor and class ID for
comparison. This information can be used to register at a server. Devices must come
prepared with their respective IDs and an endpoint or some other service where they can
register. By for instance POSTing their IDs and current firmware version to a specific
server endpoint, a server can create a profile based on the class of the device, enabling
server initiated communication in the future (pushing updates).

There are alternatives to using device profiles. One alternative is not to keep profiles
on the server, but instead keep a list of known protocols implemented by devices in the
network, and when communicating with a device trying each protocol in sequence. This has
the benefit of not needing to keep a separate profile for each class of devices and map
profiles to specific devices, but also has some issues. One issue is that you still need
to keep some state of the devices on the server regarding firmware version. If the server
does not know what the status of device versions are, it cannot help a human operator
decide about deploying updates. Another drawback is that devices might implement common
protocols but have different preferences. If two devices implement some common protocols
but one of them supports hardware operations for encrypting one of the protocols, it will
prefer using that protocol with the server, whereas the other device might not. The server
will however, with no information about device preference, try the same sequence of
protocols with both devices. 

Lastly, as communication can be unreliable over these networks, the server cannot know for
sure if a device does not respond due to not understanding the protocol and therefore
dropping the packets, or if the response just got lost in transmission. It seems more
robust to keep track of which protocols devices support and conform to the preferences of
the constrained devices.

After enrollment and registration has been done, the device enters the maintenance stage.
It can be expected to last for several years in this stage, and this is where the device
receives and applies updates. As updates can change the capabilities of devices, by for
instance implementing a new transport protocol in software, devices should confirm
successful updates at the server so the server can build a new profile for that device. If
an old, inefficient protocol has been replaced by a more efficient version the server
should switch to using the new protocol with the updated device. Furthermore, the server
will act as a source of truth regarding the firmware and software versions running on the
devices. If a device successfully changes version, the server must be aware so that future
updates to that device is of the correct version. This also requires updating of a devices
profile as the device remains in use. The device will remain in the maintenance stage
until it either breaks or is taken out of service. If you are a manufacturer of IoT
devices please consider recycling or re-using devices, starting the life cycle anew.

\section{Architecture}
\label{sec:architecture}
The architecture of the update mechanism shall enable the life cycle described in the
previous section as well as align with the goals of the SUIT standard described in Section
\ref{ssec:architecture}. In the architecture there are four important actors: the device,
the firmware server, the operator, and the certificate authority. In the scenario of
symmetric encryption a certificate authority is not needed, but as mentioned in Section
\ref{sec:device-lifecycle} the thesis will discuss the architecture from the asymmetric
cryptography point of view. If using symmetric encryption, everything is the same but the
certificate authority can be ignored.

\begin{figure}
    \caption{The proposed architecture showing two modes of updating a device.}
    \label{fig:architecture}
    \begin{tikzpicture}[box/.style={rectangle,draw,minimum width=3cm, minimum height=3.2cm},
        line/.style={->,>=latex}]
        \node[box] (operator) {Operator};
        \node[box, right=3cm of operator] (server) {Server};
        \node[box, right=3cm of server] (device) {Device};
        \node[box, below=1.5cm of server] (CA) {CA};

        \draw[line] (device.south) -- node[above,sloped] 
            {0.0 Enroll} (CA.east); 
        \draw[line] ([yshift=1.5cm]device.west) -- node[above] {0.1 Register} ([yshift=1.5cm]server.east);
            \draw[line] ([yshift=0.8cm]operator.east) -- node[above] {0.3 Manifest}
            node[below] {Image} ([yshift=0.8cm]server.west);
        \draw[line] ([yshift=0.8cm]server.east) -- node[above] {0.4 Manifest and image} ([yshift=0.8cm]device.west);
        \draw[line] ([yshift=-0.8cm]device.west) -- node[above] {0.5, 1.6 Update profile} ([yshift=-0.8cm]server.east);
        
        \draw[line, bend right] (operator.north) to[out=30,in=150] node[above] {1.3 Manifest} (device.north);
        \draw[line] ([yshift=-0.8cm]operator.east) -- node[above] {1.4 Image} ([yshift=-0.8cm]server.west);
        \draw[line] ([yshift=-1.5cm]server.east) -- node[above] {1.5 Image} ([yshift=-1.5cm]device.west);
    \end{tikzpicture}
\end{figure}

Figure \ref{fig:architecture} shows the proposed architecture and the flow of an update
mechanism. The update mechanism follows the device life cycle and presents to modes of
distributing manifests and images. The steps are numbered 0.0 to 0.5 and 1.6, with the
first three steps being mutual. The modes diverge with step 0.3 and 1.3 respectively. The
common steps for updating a device is as follows:

\begin{enumerate}[label=0.\arabic*]
    \setcounter{enumi}{-1}
    \item Enrolling the device at the certificate authority. This is one half of the
            enrollment stage in the life cycle.
    \item Registering at the server. This is the second half of the enrollment stage
            in the life cycle and creates a profile for the device on the server.
    \item The operator prepares a manifest and signs it.
\end{enumerate}

At this point the two modes diverge in how they proceed. Distributing the manifest and
image together works as follows:

\begin{enumerate}[label=0.\arabic*]
    \setcounter{enumi}{2}
    \item The manifest and image are both signed and sent to the server. If the image
            is small enough it can be attached as a payload to the manifest itself.
    \item The signed manifest and image are sent to the device. If sent in different
            packets, the manifest can be sent first and the image after the device confirms the
            manifest is correct. The device can either query the server for an update, or
            the server can push the update to the device.
    \item The device confirms the update if successful and causes the server to update
            its device profile.
\end{enumerate}

If the manifest and image are distributed separately the following steps occur instead of
steps 0.3-0.5:

\begin{enumerate}[label=1.\arabic*]
    \setcounter{enumi}{2}
    \item The signed manifest is sent directly from the operator to the device.
    \item The signed image is sent to the server.
    \item The signed image is sent to the device. This can be done either by the
        device querying the server following a successful manifest parsing, or by the server
        pushing the image to the device.
    \item The device confirms the update if successful and causes the server to update
    its device profile.
\end{enumerate}

Whether the manifest and image are distributed together or separately, they are both
signed by the operator in order to achieve end-to-end security between the operator and
device. The server should not be able to decrypt the payloads of manifest and image as
that could enable for instance MITM attacks. The device is to only accept manifests from
either the operator or server, and images from the server. This raises two important
questions: who can act as an operator, and who can act as a server?

\subsection{Who Is a Server?}
\label{ssec:who-is-a-server}
% Server: dedicated server or other device in network. Authorized by keys, needs to be on
% devices? Single point of truth for device profiles and versions. Can be used as storage
% for images being deployed later on.
Servers are responsible for transporting manifests and images to the devices. Upon
enrollment devices register at a server and the server creates a profile for that device.
The profile contains the vendor and class IDs and firmware version of the device. This
allows the server to know through which protocols the device can be contacted and which
version it should be updated to. Operators, discussed in the next section, send signed
manifests and images to servers, and can query them for device status.

Devices should contain a list of servers and by trusting the certificate authority, they
can validate those machines certificates after being enrolled. A server is thus any
machine that is enrolled, has a valid certificate, and is included in the devices list of
servers. The reasoning behind this decision is that a standard solution for updates should
not assume the topology of an IoT network. The server may be a machine acting as a proxy
between a traditional network with the operator and a constrained network with IoT
devices. The server could also be located entirely within the constrained network and be
contacted through a proxy. The server could be located entirely within the traditional
network and use a proxy to communicate with devices. 

% TODO: Does a device need one keypair for each server? Does it need one keypair for all servers?
A device could also be aware of several servers, and different devices can be mapped to
different servers. This can make device management easier as certain classes of devices
can be handled by certain servers. By allowing a device to receive updates from several
servers, the update mechanism architecture displays a form of robustness. If one server is
a machine located entirely within the constrained network and the connection between that
server and the operator is severed, updates can still be distributed through other
servers. If devices are pulling updates, they can query the servers in order of their list
of servers. If updates are pushed, devices keep the connection with the server pushing the
update.

Which machines acting like servers can be boiled down to a few important points no matter
the topology of choice:

\begin{itemize}
    \item The server is enrolled and has a valid certificate
    \item The server is included in a devices list of servers (meaning it is authorized to
            transport updates to the device)
    \item An operator can reach the server and is authorized by the server to upload updates
\end{itemize}

\subsection{Who Is an Operator?}
\label{ssec:who-is-an-operator}
% Operator: human making decisions about updates. Crafts/generates manifests, signs
% manifests and images. Authorized by keys, needs to be on device? Can send manifest for
% ease of operations, cannot send images because of size constraints/single point of
% truth/proxying? How does an operator send manifests, get profile from server? Must get
% profiles anyways for versioning.

\end{document}