\documentclass[0-thesis.tex]{subfiles}
\begin{document}
The previous sections proposed the update architecture of the thesis, its key components,
and security considerations such as identity and access control. This section will discuss
a prototype implementation of the architecture as well as a manifest generator. 

\subsection{Manifest Implementation}
\label{ssec:manifest-implementation}
As manifests contain certain information difficult for humans to provide such as
monotonically increasing sequence numbers and hash digests, a manifest generator was
created to help test the prototype \parencite{manifest-generator}. It is a Python script
which accepts information about vendor and class namespace, version, image file, and
associated URL in order to generate and format a manifest both in JSON and CBOR. The
outputted manifest follows the format specified in Section~\ref{ssec:manifest-format} and
features all required fields although some left blank. It is a bare-bones manifest
containing only the required information for a singular, monolithic update. There are no
dependencies or options specified.

The manifest is a JSON map featuring the required elements shown in
Figure~\ref{fig:manifest-format}. The elements are in the same order as in the figure but
the names of the fields have been substituted for integers in order to save space.
Table~\ref{tab:manifest-substitution} shows this mapping. Repeatable structures such as
preconditions are described as a nested array of maps, where each map in the array
constitutes one instance of such an element. The keys in these nested maps are also mapped
to integers as in the main manifest structure, but resetting the counter for each map. The
mapping of keys in nested structures is shown in Table~\ref{tab:nested-substitution}. The
example manifest used can be found in Appendix~\ref{app:manifest}. 

\begin{longtable}[]{@{}ll@{}}
    \caption{Mapping manifest elements to integers as keys in the JSON manifest.}
    \label{tab:manifest-substitution}\\
    \toprule
    Element Name & Mapped To\tabularnewline
    \midrule
    \endhead
    versionID & 0\tabularnewline
    sequenceNumber & 1\tabularnewline
    preConditions & 2\tabularnewline
    postConditions & 3\tabularnewline
    contentKeyMethod & 4\tabularnewline
    payloadInfo & 5\tabularnewline
    precursorImage & 6\tabularnewline
    dependencies & 7\tabularnewline
    options & 8\tabularnewline
    \bottomrule
\end{longtable}

\begin{longtable}[]{@{}ll@{}}
    \caption{Mapping elements in nested structures to integers.}
    \label{tab:nested-substitution}\\
    \toprule
    Element Name (corresponding structure) & Mapped To\tabularnewline
    \midrule
    \endhead
    type (conditions and options) & 0\tabularnewline
    value (conditions and options) & 1\tabularnewline
    \bottomrule
    URL (URL/digest pair) & 0\tabularnewline
    digest (URL/digest pair) & 1\tabularnewline
    \bottomrule
    format (payload info) & 0\tabularnewline
    size (payload info) & 1\tabularnewline
    storage (payload info) & 2\tabularnewline
    URL/digest pair (payload info) & 3\tabularnewline
    \bottomrule
\end{longtable}

In the client code, the manifest is received as a string, still formatted as JSON. In
order to parse it, structs resembling the format of the manifest were created, see
Listing~\ref{lst:manifest}. The base manifest structure contains values of version ID,
sequence number, and content key method alongside pointers to other parts of the manifest.
These parts, represented as repeated maps in the JSON string, are implemented as linked
lists in order to store an arbitrary amount of such structures. With the example manifest
shown in Appendix~\ref{app:manifest}, the base manifest structure would for instance
contain a two element long linked list of preconditions (vendor ID then class ID). Certain
manifest elements, namely precursor image, dependencies, and URL/digest pair in payload
info convey the same kind of information but are separated into different lists because of
the differences in semantics.

\begin{lstlisting}[language=manifest, caption={The client manifest implementation}, label=lst:manifest]
    typedef struct manifest_s {
        uint8_t versionID;
        uint32_t sequenceNumber;
        struct condition_s *preConditions;
        struct condition_s *postConditions;
        uint8_t contentKeyMethod;
        struct payloadInfo_s *payloadInfo;
        struct URLDigest_s *precursorImage;
        struct URLDigest_s *dependencies;
        struct option_s *options;
    } manifest_t;

    typedef struct condition_s {
        int8_t type;
        char *value;
        struct condition_s *next;
    } condition_t;

    typedef struct payloadInfo_s {
        uint8_t format;
        uint32_t size;
        uint8_t storage;
        struct URLDigest_s *URLDigest;
    } payloadInfo_t;

    typedef struct URLDigest_s {
        char *URL;
        char *digest;
        struct URLDigest_s *next;
    } URLDigest_t;

    typedef struct option_s {
        int8_t type;
        char *value;
        struct option_s *next;
    } option_t;
\end{lstlisting}


\subsection{Prototype Implementation}
\label{ssec:prototype-implementation}
The prototype used in the thesis is developed in order to measure the efficiency of
transport during an update procedure as well as serving as a source of inspiration for
other implementers. It consists of a server and a client both implemented in Contiki-NG,
thus written purely in C. The server is implemented in Contiki-NG as a proof-of-concept
that a more capable IoT device could be used as an update server. The prototype uses a
pull model meaning the client initiates the update procedure and the server responds with
a corresponding resource. All traffic is sent via CoAPs, encrypted by DTLS. Certificate
support in Contiki-NG was at the time of implementation missing, thus pre-shared keys were
used instead.

The server implements three resources mapped to the endpoints update/register,
update/manifest, and update/image. The register resource listens to POST requests and upon
a request extracts the vendor id, class id, and version number sent by the client and
creates a profile file before answering with message code 2.01 CREATED. Other information
can be put into the profile, such as choice of protocol and IP address, in order to reach
the device again. For the prototype, such information is not needed, and the file is
created just as proof-of-concept.

In a real deployment with different devices, thus different manifests, the manifest
resource is responsible for picking a suitable manifest based on the information in the
device's profile. For the prototype only one manifest is used which is hard-coded into the
manifest resource. Upon a GET request the manifest is encrypted through COSE and sent
using CoAP's block option. The entire manifest is encrypted at once and the ciphertext
sent block by block. Since the manifest is relatively small it is easy to allocate buffers
large enough to encrypt the entire manifest at once, leading to smaller overhead. Ideally
the manifest would be signed with COSE instead of encrypted, but COSE implementations in
Contiki-NG are limited, at the time of development signing was not available and
implementing it would prove too time consuming. Encrypting the manifest requires a bit
more information than signing, such as a nonce and Additional Authenticated Data. These
are hard coded in both client and server, alongside the key.

Lastly, the image resource generates data in a deterministic manner, COSE encrypts it
block by block, and sends through CoAP's block option. Ideally this data would also be
signed instead of encrypted. The reasoning for encrypting the data block by block is that
should a large amount of data be sent, such as a firmware image, it would be difficult to
allocate buffers large enough to process the entire image at once, both for the server and
client. Instead it is encrypted and decrypted blockwise. While this solves the issue of
encrypting and decrypting large amounts of data, it introduces a larger overhead as each
32 byte block sent will contain 24 bytes of data and 8 bytes of tag for validation. The
manifest resource only introduces this overhead once while the image resource introduces
it once per block.

The client performs three remote calls to the server, one call to each resource, and
performs some additional operations related to the manifest and image data. First a POST
request is sent to the update/register endpoint. The client does not need to act upon the
response. Afterwards a GET request is sent to the update/manifest endpoint. The manifest
callback buffers the manifest into memory block by block and when the callback finishes
decrypts the entire manifest at once. After decryption it is parsed into structs shown in
the previous section and pre-conditions are checked. If any other fields that need
checking before proceeding, such as optional pre-directives, they are also checked now. 

If the client deems the manifest to be correct it uses the URL found in the
payloadInfo->URLDigest struct to call upon that endpoint. In this case it is update/image,
but could be any endpoint hosting a resource specific for that class of device. The image
callback decrypts each block and updates a SHA-256 context with the plaintext data. After
transfer of all image data, the checksum calculation is carried out to generate a checksum
of the image data. This concludes the transfer of the update.
Figure~\ref{fig:client-server-interaction} summarizes the interactions between client and
server.

% TODO: Consider placement of this figure. Maybe earlier, before the explaining text?
\begin{figure}[h!]
    \caption{The interactions of client and server during an update procedure.}
    \label{fig:client-server-interaction}
    \includegraphics{images/client-server-sequence.pdf}
\end{figure}

\subsection{Summary}
\label{ssec:implementation-summary}
This chapter has presented a DTLS/CoAPs prototype of the proposed architecture. The
prototype is made up of a client and a server developed in Contiki-NG, as well as an
instantiation of the manifest format presented in Section~\ref{ssec:manifest-format}. The
prototype is based on the pull model, meaning it is client initiated, and goes through the
steps of registering, receiving a manifest, and receiving image data. It uses COSE
encryption and pre-shared DTLS keys and COSE signing and certificate support was at the
time not available in Contiki-NG. It also does not use tokens as it represents a
monolithic and not differential update. % TODO: This part assumes (by choice of words) tokens are optional
The next chapter will evaluate the architecture in a qualitative manner and the prototype
in a quantitative manner in order to see if it fulfills the goals of SUIT and
contextualize the implementation.

\end{document}