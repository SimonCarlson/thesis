 \documentclass[0-thesis.tex]{subfiles}

\begin{document}
This chapter presents the theoretical background needed to understand the thesis. Section
\ref{sec:network} introduces the networks protocols used and motivates their use over
other protocols in an IoT context. Section \ref{sec:hardware} presents the target hardware
for this thesis. The following section, Section \ref{sec:suit} presents and explains the
SUIT architecture and information model and their respective requirements as formulated by
the IETF. Section \ref{ssec:contiki-ng} presents the Contiki-NG operating system which the
updating mechanism will be developed for, and finally \ref{ssec:pki} introduces the PKI
which is an underlying assumption for the thesis.

\section{IoT Network Stack}
\label{sec:network}
Network protocols in IoT networks operate under different circumstances compared to
traditional computer networks. Networks in IoT are defined by their low power
requirements, low reliability, and low computational performance on edge devices. This
posts some constraints on the protocols used as they must be suitable for use in IoT
networks. One of the most widely used network protocol stacks today in traditional
networks is the TCP/IP stack. It uses TCP as a transportation protocol, usually with TLS
for security, and a common application layer protocol is HTTP. TCP is however poorly
suited for IoT networks as it is a connection based, stateful protocol which tries to
ensure the guaranteed delivery of packets in the correct order. There is also advanced
congestion control mechanisms in TCP which are hard to apply on low-bandwitch, unreliable
networks.

IoT networks on the other hand often utilize UDP for transport. UDP is also an IP-based
protocol which grants some interoperability with traditional networks while performing
better in IoT contexts. As TLS is based on the same assumptions TCP does it is unsuitable
for UDP networks. Instead there is DTLS, which is a version of TLS enhanced for use in
datagram oriented protocols such as UDP. HTTP can be used over UDP for the application
layer, but as HTTP is encoded in human readable plaintext it is unecessarily wordy and not
optimal for constrained networks. Instead, CoAP is a common protocol for the application
layer in IoT networks. This chapter briefly explains UDP, DTLS, and CoAP and explains why
they are suitable for IoT networks.

% Picture comparing network stacks?

\subsection{UDP}
\label{ssec:udp}
UDP is a stateless and asynchronous transfer protocol for IP \parencite{rfc768}. It does
not provide any reliability mechanisms but is instead a best-effort protocol. It also does
not guarantee delivery of messages. For general purposes in unconstrained environments TCP
is usually the favored transport protocol as it is robust and reliable, but in
environments where resources are scarce and networks unreliable, a stateful protocol like
TCP could face issues. Since TCP wants to ensure packet delivery, it will retransmit
packages generating a lot of traffic and processing required for a receiver. Also, if the
connection is too unstable TCP will not work at all since it can no longer guarantee the
packets arrive. The best-effort approach of UDP is favorable in these situations, in
addition to UDP being a lightweight protocol.

Figure \ref{fig:udp-header} shows the UDP header format. The source port is optional, the
length denotes the length of the datagram (including the header), and the checksum is
calculated on a pseudo-header constructed from both the IP header, UDP header, and data.

\begin{figure}
    \begin{bytefield}[bitformatting=\small, bitwidth=1.1em]{32}        
        \bitheader{0-31}\\
        \bitbox{16}{Source port} & \bitbox{16}{Destination port}\\
        \bitbox{16}{Length} & \bitbox{16}{Checksum}\\
        \bitbox[tlb]{32}{Data octets $\dots$}\\
    \end{bytefield}
    \caption{The UDP header format.}
    \label{fig:udp-header}
\end{figure}

\subsection{DTLS}
\label{ssec:dtls}
DTLS is a protocol which adds privacy to datagram protocols like UDP \parencite{rfc6347}.
The protocol is designed to prevent eavesdropping, tampering, or message forgery. DTLS is
based on TLS, a similar protocol for stateful transport protocols such as TCP, which would
not work well on unreliable networks. The main issues with using TLS over unreliable
networks is that TLS decryption is dependant on previous packets, meaning decryption of a
packet would fail if the previous packet was not received, as well as the TLS handshake
procedure assumes all handshake messages are delivered reliably.

DTLS solves this by banning stream ciphers, effectively making decryption an independent
operation between packets, as well as adding explicit sequence numbers. Furthermore, DTLS
supports packet retransmission, reordering, as well as fragmenting DTLS handshake messages
into several DTLS records. These mechanisms makes the handshake process feasible over
unreliable networks. By splitting messages into different DTLS records, fragmentation at
the IP level can be avoided since a DTLS record is guaranteed to fit an IP datagram.
% Why is avoiding IP-level fragmentation wanted?

Listing \ref{lst:dtls-plain} shows the DTLS record structure. It contains a TLS 1.2 type
field, a version field which for DTLS 1.2 is 254.253, an epoch counter that is incremented
for each cipher state change, an increasing sequence number (unique to each epoch), a
length field and a fragment field containing the application data \parencite{rfc5246}.
These fields, with the exception of the epoch and sequence number, are the same 
as in TLS 1.2 \parencite{rfc6347}.

\lstset{language=C}
\begin{lstlisting}[caption={The DTLS plaintext record structure.}, label={lst:dtls-plain}]
    struct {
        ContentType type;
        ProtocolVersion version;
        uint16 epoch;
        uint48 sequence_number;
        uint16 length;
        opaque fragment[DTLSPlaintext.length];
    } DTLSPlaintext;
\end{lstlisting}

\begin{lstlisting}[caption={The DTLS ciphertext record structure.}, label={lst:dtls-cipher}]
    struct {
        ContentType type;
        ProtocolVersion version;
        uint16 epoch;
        uint48 sequence_number;
        uint16 length;
        select (CipherSpec.cipher_type) {
            case block: GenericBlockCipher;
            case aead: GenericAEADCipher;
        } fragment;
    } DTLSCiphertext;
\end{lstlisting}

TLS records are compressed and then encrypted, the same holds for DTLS. When initating
contact, a compression algorithm is chosen. The DTLSPlaintext is then compressed into a
DTLSCompressed record, with similar structure to Listing \ref{lst:dtls-plain}.The
compressed record is encrypted into a DTLSCiphertext record whose structure is shown in
Listing \ref{lst:dtls-cipher}. Note that since DTLS disallows stream ciphers they are not
an option in the encrypted fragment, whereas in TLS they are.

In order to communicate via TLS and DTLS, a handshake has to be carried out. The handshake
establishes parameters such as protocol version, cryptographic algorithms, and shared
secrets. The TLS handshake involves hello messages for establishing algorithms, exchanging
random values, and checking for earlier sessions. Then cryptographic parameters are shared
in order to agree on a shared premaster secret. The parties authenticate each other via
public key encryption, generate a shared master secret based on the premaster secret, and
finally verifies that their peer has the correct security parameters. The DTLS handshake
adds to this a stateless cookie exchange to prevent DoS attacks, some modifications to the
handshake header to make communication over UDP possible, and retransmission timers since
the communication is unreliable. Otherwise the DTLS handshake is as the TLS handshake.

\subsection{CoAP}
\label{ssec:coap}
CoAP is an application layer protocol designed to be used by constrained devices over networks
with low throughput and possibly high unreliability for machine-to-machine communication
\parencite{rfc7252}. While designed for constrained networks, a design feature of CoAP is
how it is easily interfaced with HTTP so that communication over traditional networks can
be proxied. Furthermore, CoAP is a REST based protocol utilizing application endpoints, a
subset of standardised request/response codes, URIs, and MIME types \parencite{rest}.
Additionally CoAP offers features such as multicast support, asynchronous messages, low
header overhead, and UDP and DTLS bindings which are all suitable for constrained
environments.

As CoAP is usually implemented on top of UDP, communication is stateless and asynchronous.
For this reason CoAP defines four message types: Confirmable, Non-confirmable,
Acknowledgement, and Reset. These message types are combined with a subset of HTTP method
codes. Confirmable messages must be answered with a corresponding Acknowledgement, this
provides one form of reliability over an otherwise unreliable channel. Non-confirmable
messages do not require an Acknowledgement and thus act asynchronously. Reset messages are
used when a recipient is unable to process a Non-confirmable message. Confirmable,
Non-confirmable, and Acknowledgement messages all use Message IDs in order to detect
duplicate messages in case of retransmission.

Since CoAP is based on unreliable means of transport, there are some lightweight
reliability and congestion control mechanics in CoAP. Message IDs allows for detection of
duplicate messages and tokens allow asynchronous requests and responses be paired
correctly. There is also a retransmission mechanic with an exponential back-off timer for
Confirmable messages so that lost Acknowledgements does not cause a flood of
retransmissions.

A feature of CoAPs messaging model is the piggybacked respones. If a response to a
Confirmable or Non-confirmable request is immediately available and fits in the
Acknowledgement, the response itself can be delivered with the Acknowledgement. If the
response is not available, a recepient can respond with an empty Acknowledgement and later
send a Confirmable message containing the response. Requests use the GET, PUT, POST, and
DELETE methods in a manner that is very similar but not identical to HTTP.

Figure \ref{fig:coap} shows the CoAP message format. The 2-bit version (Ver) field
indicates the CoAP version, which at time of writing is 1 (01 in binary). The 2-bit type
(T) field determines the type of message (Confirmable, Non-confirmable, Acknowledgement,
Reset). Token length (TKL) indicates the length of the Token field which can vary between
zero to eight bytes. The 8-bit code field carries which method code the message carries
and can be further broken into a 3-bit class and 5-bit detail. The class can indicate a
request (0), a success response (2), a client error response (4), or a server error
response (5), with the detail further specifying the status of the message. The message ID
is a 16-bit integer used to detect duplicate message and to match Acknowledgement or
Resets with their initiating requests.

Following the header is the zero to eight bit Token value, which in turn is follow by zero
or more Options. Lastly comes the optional payload, which if present is prefixed by a payload
marker (0xFF). The length of the payload is dependant on the carrying protocol, which in
this thesis will be DTLS. The length of the payload is calculated depending on the size of
the CoAP header, token, and options as well as maximum DTLS record size.

% Mention header structure
\begin{figure}
    \begin{bytefield}[bitformatting={\small}, bitwidth=1.1em]{32}
        \bitheader{0-31}\\
        \bitbox{2}{Ver} & \bitbox{2}{T} & \bitbox{4}{TKL} & \bitbox{8}{Code}
        & \bitbox{16}{Message ID}\\
        \bitbox[ltb]{32}{Token (if any, TKL bytes) $\dots$}\\
        \bitbox[ltb]{32}{Options (if any) $\dots$}\\
        \bitbox{8}{1 1 1 1 1 1 1 1} & \bitbox[ltb]{24}{Payload (if any) $\dots$}
    \end{bytefield}
    \caption{The CoAP message format.}
    \label{fig:coap}
\end{figure}

Since firmware images can be relatively large their size needs to be handled during
transportation. UDP and DTLS only supports fragmentation which can be problematic in
unreliable networks. To remedy this CoAP supports block-wise transfers
\parencite{rfc7959}. A Block option allows stateless transfer of a large file separated in
different blocks. Each block can be individually retransmitted and by using monotonically
increasing block numbers, the blocks can be reassembled. The size of blocks can also be
negotiated between server and client meaning they can always find a suitable block size,
making the mechanism quite flexible.

\section{Hardware/Firefly}
\label{sec:hardware}
% Talk to Joel

\section{SUIT}
\label{sec:suit}
The IETF SUIT (Software Updates for Internet of Things) working group aims to define a
firmware update solution that is interoperable and non-proprietary \parencite{suit}. The
solution shall be usable on Class 1 devices as defined in RFC 7228. These devices feature
~10 KiB of RAM and ~100 KiB code size, which makes it suitable for this thesis
\parencite{rfc7228}. The solution may be applied to more powerful devices. The working
group does not however try to define new transport or discovery mechanism, making their
proposal angostic of any particular technology. As of writing, they have published two
documents concerning the architecture and information model of a firmware update
mechanism.

\subsection{Architecture}
\label{ssec:architecture}
There is an Internet Draft by the SUIT group focusing on the firmware update architecture
\parencite{suit-architecture}. This draft describes the goals and requirements of the
architecture, although makes no mention of any particular technology. The overarching
goals of the update process is to thwart any attempts to flash unauthorized, possibly
malicious firmware images as well as protecting the firmware image's confidentiality.
These goals reduces the possibility of an attacker either getting control over a device or
reverse engineering a malicious but valid firmware image as an attempt to mount an attack.

In order to accept an image and update itself, a device must be presented with certain
information. Several decisions must be made before updating and the information comes in
form of a manifest. The next section will describe the requirements posted upon this
manifest in more detail. The manifest helps the device make important decisions such
as if it trusts the author of the new image, if the image is intact, if the image is
applicable, where the image should be stored and so on. This in turns means the device
also has to trust the manifest itself, and that both manifest and update image must be
distributed in a safe and trusted architecture. The draft \parencite{suit-architecture}
presents ten qualitative requirements this architecture should have:
\begin{itemize}
    \item Agnostic to how firmware images are distributed:\\
            As this thesis aims to implement a prototype of an update mechanism, some
            choices about technology has to be done. This will realistically mean only a
            subset of the SUIT standard will be implemented as certain parts of the
            standard is not applicable. The proposed network stack uses UDP, DTLS, and
            CoAP for transportation and the target devices are Firefly devices running
            the Contiki-NG operating system.
    \item Friendly to broadcast delivery:\\
            Broadcasting will not be of main concern in this thesis.
    \item Use state-of-the-art security mechanisms:\\
            The SUIT standard assumes a PKI is in place. RISE has previously developed a
            PKI suitable for IoT, this PKI is an underlying assumption for the thesis. The
            PKI will allow for signing of the update manifest and firmware image.
    \item Rollback attacks must be prevented:\\
            The manifest will contain metadata such as monotonically increasing sequence
            numbers and best-before timestamps to avoid rollback attacks.
    \item High reliability:\\
            This is an implementation requirement and depends heavily on the hardware of
            the target device. % Mention possibilities for the specific bootloader of Firefly?
    \item Operate with a small bootloader:\\
            This is also an implementation requirement. % Same as above?
    \item Small parser:\\
            It must be easy to parse the fields of the update manifest as large parser can
            get quite complex. Validation of the manifest will happen on the constrained
            devices which further motivates a small parser and thus less complex
            manifests.
    \item Minimal impact on existing firmware formats:\\
            The update mechanism itself must not make assumptions of the current format of
            firmware images, but be able to support different types of firmware images.
    \item Robust permissions:\\
            This requirement is directed towards the administration of firmware updates
            and how different roles interact with the devices. The thesis will not
            consider any infrastructure outside of transporting manifest and image and
            applying the update such as device management, but will consider authorisation
            of parties through techniques like signing.
    \item Operating modes:\\
            The draft presents three broad modes of updates: client-initiated updates,
            server-initiated updates, and hybrid updates, where hybrids are mechanisms
            that require interaction between the device and firmware provider before
            updating. The thesis will look into all three of these broad classes.
\end{itemize}

An example architecture encompassing a device, a firmware update author, a firmware
server, a network operator and a device operator is presented in the draft. The author and
the device ineract with the firmware server in order to communicate firmware updates and
possibly manifests. The communication of the device as well as the updating itself is the
concern of this thesis. The draft also presents device management for the device operator
so that it is possible to track the state of updates in a network. Device management is
considered out of scope for this thesis.

The distribution of manifest and firmware image is also discussed, with a couple of
options being possible. The manifest and image can be distributed together to a firmware
server. The device then recieves the manifest either via pulling or pushing and can
subsequently download the image. Alternatively, the manifest itself can be directly sent
to the device without a need of a firmware server, while the firmware image is put on the
firmware server. After the device has receieved the lone manifest through some method, the
firmware can be downloaded from the firmware server. The SUIT architecture does not
enforce a specific method to be used when delivering the manifest and firmware, but states
that an update mechanism must support both types.

% TODO: Images showing the differences of distribution

\subsection{Information Model}
\label{ssec:information-model}
The corresponding Internet Draft for the information model presents the information needed
in the manifest to secure a firmware update mechanism \parencite{suit-information-model}.
It also presents threats, classifies them according to the STRIDE model, and presents
security requirements that map to the threats \parencite{stride}. Furthermore it presents
use cases and maps usability requirements to the use cases. Finally it presents the
proposed elements of the manifest. Since the thesis makes a choice about specific
technologies to use, not all use cases, usability requirements, and manifest elements are
deemed necessary. The threats and security requirements however are. Note that the
information model does not discuss threats outside of transporting the updates, such as
physical attacks.

Table \ref{tab:threats-to-requirements} shows how the security threats map to security
requirements along with a brief description of each threat and their class. All threats,
MFT1 - MFT12, are considered relevant for this project and are what the SUIT information
model attempts to prevent. The security requirements, with brief descriptions, and how
they map to the manifest elements can be found in Table
\ref{tab:requirements-to-elements}. Finally, Table \ref{tab:elements-to-requirements}
shows the proposed manifest elements, their status, and which security requirements each
element implements.

\begin{longtable}[]{@{}llll@{}}
    \caption{Mapping security threats to security requirements.}
    \label{tab:threats-to-requirements}\\
    \toprule
    \begin{minipage}[b]{0.08\columnwidth}\raggedright\strut Threat\strut \end{minipage} &
    \begin{minipage}[b]{0.44\columnwidth}\raggedright\strut Description\strut
    \end{minipage} & \begin{minipage}[b]{0.30\columnwidth}\raggedright\strut Threat
    Model\strut \end{minipage} & \begin{minipage}[b]{0.20\columnwidth}\raggedright\strut
    Mitigated By\strut \end{minipage}\tabularnewline
    \midrule
    \endhead
    \begin{minipage}[t]{0.05\columnwidth}\raggedright\strut MFT1\strut \end{minipage} &
    \begin{minipage}[t]{0.44\columnwidth}\raggedright\strut Old Firmware\strut
    \end{minipage} & \begin{minipage}[t]{0.30\columnwidth}\raggedright\strut Elevation of
    Privilege\strut \end{minipage} &
    \begin{minipage}[t]{0.09\columnwidth}\raggedright\strut MFSR1\strut
    \end{minipage}\tabularnewline
    \begin{minipage}[t]{0.05\columnwidth}\raggedright\strut MFT2\strut \end{minipage} &
    \begin{minipage}[t]{0.44\columnwidth}\raggedright\strut Mismatched Firmware\strut
    \end{minipage} & \begin{minipage}[t]{0.30\columnwidth}\raggedright\strut Denial of
    Service\strut \end{minipage} & \begin{minipage}[t]{0.09\columnwidth}\raggedright\strut
    MFSR2\strut \end{minipage}\tabularnewline
    \begin{minipage}[t]{0.05\columnwidth}\raggedright\strut MFT3\strut \end{minipage} &
    \begin{minipage}[t]{0.44\columnwidth}\raggedright\strut Offline device + old
    firmware\strut \end{minipage} &
    \begin{minipage}[t]{0.30\columnwidth}\raggedright\strut Elevation of Privilege\strut
    \end{minipage} & \begin{minipage}[t]{0.09\columnwidth}\raggedright\strut MFSR3\strut
    \end{minipage}\tabularnewline
    \begin{minipage}[t]{0.05\columnwidth}\raggedright\strut MFT4\strut \end{minipage} &
    \begin{minipage}[t]{0.44\columnwidth}\raggedright\strut The target device
    misinterprets the type of payload\strut \end{minipage} &
    \begin{minipage}[t]{0.30\columnwidth}\raggedright\strut Denial of Service\strut
    \end{minipage} & \begin{minipage}[t]{0.09\columnwidth}\raggedright\strut MFSRa\strut
    \end{minipage}\tabularnewline
    \begin{minipage}[t]{0.05\columnwidth}\raggedright\strut MFT5\strut \end{minipage} &
    \begin{minipage}[t]{0.44\columnwidth}\raggedright\strut The target device installs the
    payload to the wrong location\strut \end{minipage} &
    \begin{minipage}[t]{0.30\columnwidth}\raggedright\strut Denial of Service\strut
    \end{minipage} & \begin{minipage}[t]{0.09\columnwidth}\raggedright\strut MFSR4b\strut
    \end{minipage}\tabularnewline
    \begin{minipage}[t]{0.05\columnwidth}\raggedright\strut MFT6\strut \end{minipage} &
    \begin{minipage}[t]{0.44\columnwidth}\raggedright\strut Redirection\strut
    \end{minipage} & \begin{minipage}[t]{0.30\columnwidth}\raggedright\strut Denial of
    Service\strut \end{minipage} & \begin{minipage}[t]{0.09\columnwidth}\raggedright\strut
    MFSR4c\strut \end{minipage}\tabularnewline
    \begin{minipage}[t]{0.05\columnwidth}\raggedright\strut MFT7\strut \end{minipage} &
    \begin{minipage}[t]{0.44\columnwidth}\raggedright\strut Payload Verification on
    Boot\strut \end{minipage} & \begin{minipage}[t]{0.30\columnwidth}\raggedright\strut
    Elevation of Privilege\strut \end{minipage} &
    \begin{minipage}[t]{0.09\columnwidth}\raggedright\strut MFSR4d\strut
    \end{minipage}\tabularnewline
    \begin{minipage}[t]{0.05\columnwidth}\raggedright\strut MFT8\strut \end{minipage} &
    \begin{minipage}[t]{0.44\columnwidth}\raggedright\strut Unauthenticated Updates\strut
    \end{minipage} & \begin{minipage}[t]{0.30\columnwidth}\raggedright\strut Elevation of
    Privilege\strut \end{minipage} &
    \begin{minipage}[t]{0.09\columnwidth}\raggedright\strut MFSR5\strut
    \end{minipage}\tabularnewline
    \begin{minipage}[t]{0.05\columnwidth}\raggedright\strut MFT9\strut \end{minipage} &
    \begin{minipage}[t]{0.44\columnwidth}\raggedright\strut Unexpected Precursor
    Image\strut \end{minipage} & \begin{minipage}[t]{0.30\columnwidth}\raggedright\strut
    Denial of Service\strut \end{minipage} &
    \begin{minipage}[t]{0.09\columnwidth}\raggedright\strut MFSR4e\strut
    \end{minipage}\tabularnewline
    \begin{minipage}[t]{0.05\columnwidth}\raggedright\strut MFT10\strut \end{minipage} &
    \begin{minipage}[t]{0.44\columnwidth}\raggedright\strut Unqualified Firmware\strut
    \end{minipage} & \begin{minipage}[t]{0.30\columnwidth}\raggedright\strut Denial of
    Service, Elevation of Privilege\strut \end{minipage} &
    \begin{minipage}[t]{0.09\columnwidth}\raggedright\strut MFSR6, MFSR8\strut
    \end{minipage}\tabularnewline
    \begin{minipage}[t]{0.05\columnwidth}\raggedright\strut MFT11\strut \end{minipage} &
    \begin{minipage}[t]{0.44\columnwidth}\raggedright\strut Reverse Engineering Of
    Firmware Image for Vulnerability Analysis\strut \end{minipage} &
    \begin{minipage}[t]{0.30\columnwidth}\raggedright\strut All classes\strut
    \end{minipage} & \begin{minipage}[t]{0.09\columnwidth}\raggedright\strut MFSR7\strut
    \end{minipage}\tabularnewline
    \begin{minipage}[t]{0.05\columnwidth}\raggedright\strut MFT12\strut \end{minipage} &
    \begin{minipage}[t]{0.44\columnwidth}\raggedright\strut Overriding Critical Manifest
    Elements\strut \end{minipage} &
    \begin{minipage}[t]{0.30\columnwidth}\raggedright\strut Elevation of Privilege\strut
    \end{minipage} & \begin{minipage}[t]{0.09\columnwidth}\raggedright\strut MFSR8\strut
    \end{minipage}\tabularnewline
    \bottomrule
\end{longtable}

\begin{longtable}[]{@{}llll@{}}
    \caption{Mapping security requirements to manifest elements.}
    \label{tab:requirements-to-elements}\\
    \toprule
    \begin{minipage}[b]{0.16\columnwidth}\raggedright\strut Security Requirement\strut
    \end{minipage} & \begin{minipage}[b]{0.29\columnwidth}\raggedright\strut
    Description\strut \end{minipage} &
    \begin{minipage}[b]{0.34\columnwidth}\raggedright\strut Implemented By\strut
    \end{minipage} & \begin{minipage}[b]{0.10\columnwidth}\raggedright\strut
    Mitigates\strut \end{minipage}\tabularnewline
    \midrule
    \endhead
    \begin{minipage}[t]{0.16\columnwidth}\raggedright\strut MFSR1\strut \end{minipage} &
    \begin{minipage}[t]{0.29\columnwidth}\raggedright\strut Monotonic Sequence
    Numbers\strut \end{minipage} & \begin{minipage}[t]{0.34\columnwidth}\raggedright\strut
    Monotonic Sequence Number\strut \end{minipage} &
    \begin{minipage}[t]{0.10\columnwidth}\raggedright\strut MFT1\strut
    \end{minipage}\tabularnewline
    \begin{minipage}[t]{0.16\columnwidth}\raggedright\strut MFSR2\strut \end{minipage} &
    \begin{minipage}[t]{0.29\columnwidth}\raggedright\strut Vendor and Device-type
    Identifiers\strut \end{minipage} &
    \begin{minipage}[t]{0.34\columnwidth}\raggedright\strut Vendor ID Condition, Class ID
    Condition\strut \end{minipage} &
    \begin{minipage}[t]{0.10\columnwidth}\raggedright\strut MFT2\strut
    \end{minipage}\tabularnewline
    \begin{minipage}[t]{0.16\columnwidth}\raggedright\strut MFSR3\strut \end{minipage} &
    \begin{minipage}[t]{0.29\columnwidth}\raggedright\strut Best-Before Timestamps\strut
    \end{minipage} & \begin{minipage}[t]{0.34\columnwidth}\raggedright\strut Best-Before
    timestamp condition\strut \end{minipage} &
    \begin{minipage}[t]{0.10\columnwidth}\raggedright\strut MFT3\strut
    \end{minipage}\tabularnewline
    \begin{minipage}[t]{0.16\columnwidth}\raggedright\strut MFSR4a\strut \end{minipage} &
    \begin{minipage}[t]{0.29\columnwidth}\raggedright\strut Authenticated Payload
    Type\strut \end{minipage} & \begin{minipage}[t]{0.34\columnwidth}\raggedright\strut
    Payload Format, Storage Location\strut \end{minipage} &
    \begin{minipage}[t]{0.10\columnwidth}\raggedright\strut MFT4\strut
    \end{minipage}\tabularnewline
    \begin{minipage}[t]{0.16\columnwidth}\raggedright\strut MFSR4b\strut \end{minipage} &
    \begin{minipage}[t]{0.29\columnwidth}\raggedright\strut Authenticated Storage
    Location\strut \end{minipage} &
    \begin{minipage}[t]{0.34\columnwidth}\raggedright\strut Storage Location\strut
    \end{minipage} & \begin{minipage}[t]{0.10\columnwidth}\raggedright\strut MFT5\strut
    \end{minipage}\tabularnewline
    \begin{minipage}[t]{0.16\columnwidth}\raggedright\strut MFSR4c\strut \end{minipage} &
    \begin{minipage}[t]{0.29\columnwidth}\raggedright\strut Authenticated Remote Resource
    Location\strut \end{minipage} &
    \begin{minipage}[t]{0.34\columnwidth}\raggedright\strut URIs\strut \end{minipage} &
    \begin{minipage}[t]{0.10\columnwidth}\raggedright\strut MFT6\strut
    \end{minipage}\tabularnewline
    \begin{minipage}[t]{0.16\columnwidth}\raggedright\strut MFSR4d\strut \end{minipage} &
    \begin{minipage}[t]{0.29\columnwidth}\raggedright\strut Secure Boot\strut
    \end{minipage} & \begin{minipage}[t]{0.34\columnwidth}\raggedright\strut Payload
    Digest, Size\strut \end{minipage} &
    \begin{minipage}[t]{0.10\columnwidth}\raggedright\strut MFT7\strut
    \end{minipage}\tabularnewline
    \begin{minipage}[t]{0.16\columnwidth}\raggedright\strut MFSR4e\strut \end{minipage} &
    \begin{minipage}[t]{0.29\columnwidth}\raggedright\strut Authenticated precursor
    images\strut \end{minipage} & \begin{minipage}[t]{0.34\columnwidth}\raggedright\strut
    Precursor Image Digest Condition\strut \end{minipage} &
    \begin{minipage}[t]{0.10\columnwidth}\raggedright\strut MFT9\strut
    \end{minipage}\tabularnewline
    \begin{minipage}[t]{0.16\columnwidth}\raggedright\strut MFSR4f\strut \end{minipage} &
    \begin{minipage}[t]{0.29\columnwidth}\raggedright\strut Authenticated Vendor and Class
    IDs\strut \end{minipage} & \begin{minipage}[t]{0.34\columnwidth}\raggedright\strut
    Vendor ID Condition, Class ID Condition\strut \end{minipage} &
    \begin{minipage}[t]{0.10\columnwidth}\raggedright\strut MFT2\strut
    \end{minipage}\tabularnewline
    \begin{minipage}[t]{0.16\columnwidth}\raggedright\strut MFSR5\strut \end{minipage} &
    \begin{minipage}[t]{0.29\columnwidth}\raggedright\strut Cryptographic
    Authenticity\strut \end{minipage} &
    \begin{minipage}[t]{0.34\columnwidth}\raggedright\strut Signature, Payload
    Digest\strut \end{minipage} & \begin{minipage}[t]{0.10\columnwidth}\raggedright\strut
    MFT8\strut \end{minipage}\tabularnewline
    \begin{minipage}[t]{0.16\columnwidth}\raggedright\strut MFSR6\strut \end{minipage} &
    \begin{minipage}[t]{0.29\columnwidth}\raggedright\strut Rights Require
    Authenticity\strut \end{minipage} &
    \begin{minipage}[t]{0.34\columnwidth}\raggedright\strut Signature\strut \end{minipage}
    & \begin{minipage}[t]{0.10\columnwidth}\raggedright\strut MFT10\strut
    \end{minipage}\tabularnewline
    \begin{minipage}[t]{0.16\columnwidth}\raggedright\strut MFSR7\strut \end{minipage} &
    \begin{minipage}[t]{0.29\columnwidth}\raggedright\strut Firmware Encryption\strut
    \end{minipage} & \begin{minipage}[t]{0.34\columnwidth}\raggedright\strut Content Key
    Distribution Method\strut \end{minipage} &
    \begin{minipage}[t]{0.10\columnwidth}\raggedright\strut MFT11\strut
    \end{minipage}\tabularnewline
    \begin{minipage}[t]{0.16\columnwidth}\raggedright\strut MFSR8\strut \end{minipage} &
    \begin{minipage}[t]{0.29\columnwidth}\raggedright\strut Access Control Lists\strut
    \end{minipage} & \begin{minipage}[t]{0.34\columnwidth}\raggedright\strut Client-side
    code (not specified in manifest)\strut \end{minipage} &
    \begin{minipage}[t]{0.10\columnwidth}\raggedright\strut MFT10, MFT12\strut
    \end{minipage}\tabularnewline
    \bottomrule
\end{longtable}

\begin{longtable}[]{@{}lll@{}}
    \caption{Manifest elements, their status, and which security requirements they implement.}
    \label{tab:elements-to-requirements}\\
    \toprule
    \begin{minipage}[b]{0.32\columnwidth}\raggedright\strut Manifest Element\strut
    \end{minipage} & \begin{minipage}[b]{0.36\columnwidth}\raggedright\strut Status\strut
    \end{minipage} & \begin{minipage}[b]{0.23\columnwidth}\raggedright\strut Implements
    Requirements\strut \end{minipage}\tabularnewline
    \midrule
    \endhead
    \begin{minipage}[t]{0.32\columnwidth}\raggedright\strut Version identifier\strut
    \end{minipage} & \begin{minipage}[t]{0.36\columnwidth}\raggedright\strut
    MANDATORY\strut \end{minipage} &
    \begin{minipage}[t]{0.23\columnwidth}\raggedright\strut
    \strut
    \end{minipage}\tabularnewline
    \begin{minipage}[t]{0.32\columnwidth}\raggedright\strut Monotonic Sequence
    Number\strut \end{minipage} & \begin{minipage}[t]{0.36\columnwidth}\raggedright\strut
    MANDATORY\strut \end{minipage} &
    \begin{minipage}[t]{0.23\columnwidth}\raggedright\strut MFSR1\strut
    \end{minipage}\tabularnewline
    \begin{minipage}[t]{0.32\columnwidth}\raggedright\strut Precursor Image Digest
    Condition\strut \end{minipage} &
    \begin{minipage}[t]{0.36\columnwidth}\raggedright\strut MANDATORY (for differential
    updates)\strut \end{minipage} &
    \begin{minipage}[t]{0.23\columnwidth}\raggedright\strut MFSR4e\strut
    \end{minipage}\tabularnewline
    \begin{minipage}[t]{0.32\columnwidth}\raggedright\strut Payload Format\strut
    \end{minipage} & \begin{minipage}[t]{0.36\columnwidth}\raggedright\strut
    MANDATORY\strut \end{minipage} &
    \begin{minipage}[t]{0.23\columnwidth}\raggedright\strut MFSR4a, MFUR5\strut
    \end{minipage}\tabularnewline
    \begin{minipage}[t]{0.32\columnwidth}\raggedright\strut Storage Location\strut
    \end{minipage} & \begin{minipage}[t]{0.36\columnwidth}\raggedright\strut
    MANDATORY\strut \end{minipage} &
    \begin{minipage}[t]{0.23\columnwidth}\raggedright\strut MFSR4b\strut
    \end{minipage}\tabularnewline
    \begin{minipage}[t]{0.32\columnwidth}\raggedright\strut Payload Digest\strut
    \end{minipage} & \begin{minipage}[t]{0.36\columnwidth}\raggedright\strut
    MANDATORY\strut \end{minipage} &
    \begin{minipage}[t]{0.23\columnwidth}\raggedright\strut MFSR4d, MFUR8\strut
    \end{minipage}\tabularnewline
    \begin{minipage}[t]{0.32\columnwidth}\raggedright\strut Size\strut \end{minipage} &
    \begin{minipage}[t]{0.36\columnwidth}\raggedright\strut MANDATORY\strut \end{minipage}
    & \begin{minipage}[t]{0.23\columnwidth}\raggedright\strut MFSR4d\strut
    \end{minipage}\tabularnewline
    \begin{minipage}[t]{0.32\columnwidth}\raggedright\strut Signature\strut \end{minipage}
    & \begin{minipage}[t]{0.36\columnwidth}\raggedright\strut MANDATORY\strut
    \end{minipage} & \begin{minipage}[t]{0.23\columnwidth}\raggedright\strut MFSR5, MFSR6,
    MFUR6\strut \end{minipage}\tabularnewline
    \begin{minipage}[t]{0.32\columnwidth}\raggedright\strut Dependencies\strut
    \end{minipage} & \begin{minipage}[t]{0.36\columnwidth}\raggedright\strut
    MANDATORY\strut \end{minipage} &
    \begin{minipage}[t]{0.23\columnwidth}\raggedright\strut MFUR3\strut
    \end{minipage}\tabularnewline
    \begin{minipage}[t]{0.32\columnwidth}\raggedright\strut Content Key Distribution
    Method\strut \end{minipage} & \begin{minipage}[t]{0.36\columnwidth}\raggedright\strut
    MANDATORY (for encrypted payloads)\strut \end{minipage} &
    \begin{minipage}[t]{0.23\columnwidth}\raggedright\strut MFSR7\strut
    \end{minipage}\tabularnewline
    \begin{minipage}[t]{0.32\columnwidth}\raggedright\strut Vendor ID Condition\strut
    \end{minipage} & \begin{minipage}[t]{0.36\columnwidth}\raggedright\strut
    OPTIONAL\strut \end{minipage} &
    \begin{minipage}[t]{0.23\columnwidth}\raggedright\strut MFSR2, MFSR4f\strut
    \end{minipage}\tabularnewline
    \begin{minipage}[t]{0.32\columnwidth}\raggedright\strut Class ID Condition\strut
    \end{minipage} & \begin{minipage}[t]{0.36\columnwidth}\raggedright\strut
    OPTIONAL\strut \end{minipage} &
    \begin{minipage}[t]{0.23\columnwidth}\raggedright\strut MFSR2, MFSR4f\strut
    \end{minipage}\tabularnewline
    \begin{minipage}[t]{0.32\columnwidth}\raggedright\strut Required Image Version
    List\strut \end{minipage} & \begin{minipage}[t]{0.36\columnwidth}\raggedright\strut
    OPTIONAL\strut \end{minipage} &
    \begin{minipage}[t]{0.23\columnwidth}\raggedright\strut MFUR7\strut
    \end{minipage}\tabularnewline
    \begin{minipage}[t]{0.32\columnwidth}\raggedright\strut Best-Before Timestamp
    Condition\strut \end{minipage} &
    \begin{minipage}[t]{0.36\columnwidth}\raggedright\strut OPTIONAL\strut \end{minipage}
    & \begin{minipage}[t]{0.23\columnwidth}\raggedright\strut MFSR3\strut
    \end{minipage}\tabularnewline
    \begin{minipage}[t]{0.32\columnwidth}\raggedright\strut Component Identifier\strut
    \end{minipage} & \begin{minipage}[t]{0.36\columnwidth}\raggedright\strut
    OPTIONAL\strut \end{minipage} &
    \begin{minipage}[t]{0.23\columnwidth}\raggedright\strut MFUR3\strut
    \end{minipage}\tabularnewline
    \begin{minipage}[t]{0.32\columnwidth}\raggedright\strut URIs\strut \end{minipage} &
    \begin{minipage}[t]{0.36\columnwidth}\raggedright\strut OPTIONAL\strut \end{minipage}
    & \begin{minipage}[t]{0.23\columnwidth}\raggedright\strut MFSR4c\strut
    \end{minipage}\tabularnewline
    \begin{minipage}[t]{0.32\columnwidth}\raggedright\strut Directives\strut
    \end{minipage} & \begin{minipage}[t]{0.36\columnwidth}\raggedright\strut
    OPTIONAL\strut \end{minipage} &
    \begin{minipage}[t]{0.23\columnwidth}\raggedright\strut MFUR1\strut
    \end{minipage}\tabularnewline
    \begin{minipage}[t]{0.32\columnwidth}\raggedright\strut Aliases\strut \end{minipage} &
    \begin{minipage}[t]{0.36\columnwidth}\raggedright\strut OPTIONAL\strut \end{minipage}
    & \begin{minipage}[t]{0.23\columnwidth}\raggedright\strut MFUR2\strut
    \end{minipage}\tabularnewline
    \begin{minipage}[t]{0.32\columnwidth}\raggedright\strut Processing Steps\strut
    \end{minipage} & \begin{minipage}[t]{0.36\columnwidth}\raggedright\strut
    \strut
    \end{minipage} & \begin{minipage}[t]{0.23\columnwidth}\raggedright\strut MFUR6\strut
    \end{minipage}\tabularnewline
    \begin{minipage}[t]{0.32\columnwidth}\raggedright\strut XIP Address\strut
    \end{minipage} & \begin{minipage}[t]{0.36\columnwidth}\raggedright\strut
    \strut
    \end{minipage} & \begin{minipage}[t]{0.23\columnwidth}\raggedright\strut MFUR8\strut
    \end{minipage}\tabularnewline
    \bottomrule
\end{longtable}

% Which manifest elements are deemed most useful/needed? Is this a question for the design
% part of the thesis?

To summarize, the SUIT information model proposes to use a signed manifest that is
distributed to each device in need of an update. The device then processes the manifest in
order to determine if the update is trusted, suitable, up to date, with many other
optional elements such as if other precursor images, special processing steps, or new URIs
to fetch the images are needed. The model does not make assumptions about technology which
is one of the reasons there are optional elements, not all of them are applicable to all
solutions. Nevertheless, the architecture and information model together provides a solid
base on which to design a secure update mechanism for IoT.

\section{Contiki-NG}
\label{ssec:contiki-ng}
% Mention primitives such as processes, timers, network stack For details on image
% flipping/bootloaders, Niclas should know
Contiki-NG is an open-source operating system for resource constrained IoT devices based
on the Contiki operating system \parencite{contiki-ng-github}, \parencite{contiki-github}.
Contiki-NG focuses on low-power communication and standard protocols and comes with
IPv6/LoWPAN, DTLS, and CoAP implementations which makes it a suitable operating system for
this thesis. Furthermore, Contiki-NG is open source and licensed under the permissive BSD
3-Clause license and targets a wide variety of boards which makes it align with SUITs goal
of creating an open standard for updating IoT devices.

\subsection{Processes, events, and memory management}
\label{ssec:process-event-memory}
Contiki-NG has a process abstraction which is built on lightweight protothreads
\parencite{protothreads}. A process is declared through a PROCESS macro and can be
automatically started after system boot or when a specific module is loaded. User-space
processes are run in a cooperative manner while kernel-space processes can preempt
user-space processes. Contiki-NGs execution model is event based, meaning processes often
yield execution until they are informed a certain event has taken place, upon which they
can act. Examples of events are timers expiring, a process being polled, or a network
packet arriving. 

Contiki-NG provides to memory allocators in addition to using static memory. They are
called MEMB and HeapMem and are semi-dynamic and dynamic, respectively. MEMB provides ways
to manage memory blocks. The memory blocks are allocated on static memory as arrays of
constant sized objects. After a memory block has been declared, it is initialized after
which objects can be allocated memory from the block. All objects allocated through the
same block have the same size. Blocks can be freed, and it is possible to check whether a
pointer resides within a certain memory block or not.

HeapMem solves the issue of dynamically allocating objects of varying sizes during runtime
in Contiki-NG. It can be used on a variety of hardware platforms, something a standard C
malloc implementation could struggle with. To allocate memory on the heap, the number of
bytes to allocate must be provided and a pointer to a contiguous piece of memory is
returned (if there is enough contiguous memory). Memory can be reallocated and deallocated
such as using a normal malloc.

\subsection{Networking in Contiki-NG}
\label{ssec:networking-contiki}


\section{PKI?}
\label{ssec:pki}
% Talk to Sam or Shahid

\end{document}