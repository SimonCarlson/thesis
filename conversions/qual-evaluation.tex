\begin{longtable}[]{@{}ll@{}}
\toprule
\begin{minipage}[b]{0.41\columnwidth}\raggedright\strut
Reqruirement\strut
\end{minipage} & \begin{minipage}[b]{0.53\columnwidth}\raggedright\strut
Motivation\strut
\end{minipage}\tabularnewline
\midrule
\endhead
\begin{minipage}[t]{0.41\columnwidth}\raggedright\strut
Agnostic to how firmware images are distributed\strut
\end{minipage} & \begin{minipage}[t]{0.53\columnwidth}\raggedright\strut
No parts of the architecture assumes a specific suite of protocols or
algorithms to ensure secure delivery of updates. The architecture does
specify the usage of technologies such as certificates, asymmetric
encryption, and authorization tokens, but these elements can be
implemented in a wide variety of ways using different protocols,
algorithms, and certificate/token types. The profiles provided in the
thesis are examples of how specific incarnations of the architecture
could look like.\strut
\end{minipage}\tabularnewline
\begin{minipage}[t]{0.41\columnwidth}\raggedright\strut
Friendly to broadcast delivery\strut
\end{minipage} & \begin{minipage}[t]{0.53\columnwidth}\raggedright\strut
The architecture itself does not limit broadcast delivery and through
correct usage of the manifest broadcasted updates will not be
incorrectly installed by devices other than the intended recipients.
Choice of technology can limit the usage of broadcasting, such as DTLS,
but this is implementation specific.\strut
\end{minipage}\tabularnewline
\begin{minipage}[t]{0.41\columnwidth}\raggedright\strut
Use state-of-the-art security mechanisms\strut
\end{minipage} & \begin{minipage}[t]{0.53\columnwidth}\raggedright\strut
The architecture is based on asymmetric cryptography using certificates
as well as access authorization through whitelists and possibly tokens
for fine-grain authorization. Choice of key algorithms and token types
will affect the cryptographic strength of the system, allowing for both
smaller and less strong keys and tokens or larger and strong keys and
tokens.\strut
\end{minipage}\tabularnewline
\begin{minipage}[t]{0.41\columnwidth}\raggedright\strut
Rollback attacks must be prevented\strut
\end{minipage} & \begin{minipage}[t]{0.53\columnwidth}\raggedright\strut
By specifying a monotonically increasing sequence number in the
manifest, devices can make sure they are installing fresh images.
Manifests are signed through COSE which ensures authenticity, an
assailant would not be able to modify a manifest and then recompute a
valid signature.\strut
\end{minipage}\tabularnewline
\begin{minipage}[t]{0.41\columnwidth}\raggedright\strut
High reliability\strut
\end{minipage} & \begin{minipage}[t]{0.53\columnwidth}\raggedright\strut
This is an implementation specific requirement, however the storage
element of the manifest aids in achieving safe storage of a new image.
After a successful update, devices are to re-register at servers. No
acknowledgment means the server knows the update still must be applied,
thus an interrupted update can be redistributed.\strut
\end{minipage}\tabularnewline
\begin{minipage}[t]{0.41\columnwidth}\raggedright\strut
Operate with a small bootloader\strut
\end{minipage} & \begin{minipage}[t]{0.53\columnwidth}\raggedright\strut
The thesis suggests to store an unencrypted image alongside its digest
for the bootloader to be minimal, only needing support for calculating a
SHA256 digest of the image at boot. All information about whether or not
to perform the update is encoded in conditions in the manifest, and can
be stored with small memory usage.\strut
\end{minipage}\tabularnewline
\begin{minipage}[t]{0.41\columnwidth}\raggedright\strut
Small parsers\strut
\end{minipage} & \begin{minipage}[t]{0.53\columnwidth}\raggedright\strut
The manifest format used in the thesis is simple, complete, and
extensible. Parsing it requires storing the key/value pairs in
predefined structures which is doable even with a very small
parser.\strut
\end{minipage}\tabularnewline
\begin{minipage}[t]{0.41\columnwidth}\raggedright\strut
Minimal impact on existing firmware formats\strut
\end{minipage} & \begin{minipage}[t]{0.53\columnwidth}\raggedright\strut
The architecture makes no assumptions about firmware formats. It is up
to the implementer to ensure the images are prepared for updating and
that the bootloader contains the necessary functionality. Transporting
the images is the same regardless of content.\strut
\end{minipage}\tabularnewline
\begin{minipage}[t]{0.41\columnwidth}\raggedright\strut
Robust permissions\strut
\end{minipage} & \begin{minipage}[t]{0.53\columnwidth}\raggedright\strut
The architecture is based on asymmetric cryptography using certificates
for identity and confidentiality, and whitelists of servers and
operators as well as tokens for authorization. Different deployments
have different security and performance requirements, the architecture
is flexible by allowing different kinds of key algorithms, encryption
schemes, and token types. Some deployments may not need tokens at all as
authentication is done on a device level and whitelists are sufficient,
white other deployments wanting to update specific pieces of a device
may need fine-grained authorization through tokens. Different devices in
the same deployment can have different authorization configurations, for
instance by accepting different types of tokens.\strut
\end{minipage}\tabularnewline
\begin{minipage}[t]{0.41\columnwidth}\raggedright\strut
Operating modes\strut
\end{minipage} & \begin{minipage}[t]{0.53\columnwidth}\raggedright\strut
The architecture supports the device initiated pull model as well as the
operator initiated push model. The update server acts as a mediator
between device and operator as well as a repository for images and
device profiles, letting the operator query the server for device
statuses and devices query for updates depending on the model
used.\strut
\end{minipage}\tabularnewline
\bottomrule
\end{longtable}

\begin{longtable}[]{@{}ll@{}}
\toprule
\begin{minipage}[b]{0.41\columnwidth}\raggedright\strut
Requirement\strut
\end{minipage} & \begin{minipage}[b]{0.53\columnwidth}\raggedright\strut
Motivation\strut
\end{minipage}\tabularnewline
\midrule
\endhead
\begin{minipage}[t]{0.41\columnwidth}\raggedright\strut
Installation instructions\strut
\end{minipage} & \begin{minipage}[t]{0.53\columnwidth}\raggedright\strut
Installation instructions or any sort of directive is not included in
the base manifest structure but can be implemented in the options field.
Directives, shown as an option in
Figure\textasciitilde{}\ref{fig:manifest-format}, could be used to
indicate decompression algorithms, preparation of bootloader etc while
other more specific instructions can be implemented either as a
directive or a new kind of option element specific to that
deployment.\strut
\end{minipage}\tabularnewline
\begin{minipage}[t]{0.41\columnwidth}\raggedright\strut
Override non-critical manifest elements\strut
\end{minipage} & \begin{minipage}[t]{0.53\columnwidth}\raggedright\strut
The proposed manifest format has the critical elements as a baseline and
the rest severed into the options field. Defining new option types
allows for new elements in the manifest, but the devices need to be
updated for their parsers and manifest checkers to be aware of the new
elements.\strut
\end{minipage}\tabularnewline
\begin{minipage}[t]{0.41\columnwidth}\raggedright\strut
Modular update\strut
\end{minipage} & \begin{minipage}[t]{0.53\columnwidth}\raggedright\strut
Supported through precursor images, dependencies, URI aliases, and the
use of authorization tokens for more granular updates. For devices with
several MCUs and applications, possibly several operating systems,
update access can be controlled through authorization tokens such that a
vendor can only update their respective part.\strut
\end{minipage}\tabularnewline
\begin{minipage}[t]{0.41\columnwidth}\raggedright\strut
Multiple authorizations\strut
\end{minipage} & \begin{minipage}[t]{0.53\columnwidth}\raggedright\strut
Enforced via operator and update server whitelists and authorization
tokens.\strut
\end{minipage}\tabularnewline
\begin{minipage}[t]{0.41\columnwidth}\raggedright\strut
Multiple payload formats\strut
\end{minipage} & \begin{minipage}[t]{0.53\columnwidth}\raggedright\strut
The architecture and manifest format does not make assumptions about the
payload formats. The mappings of formats to keys in the manifest is
deployment specific and whatever formats a deployment might opt to use
can be represented in the manifest together with a SHA256 digest of said
format.\strut
\end{minipage}\tabularnewline
\begin{minipage}[t]{0.41\columnwidth}\raggedright\strut
Prevent confidential information disclosure\strut
\end{minipage} & \begin{minipage}[t]{0.53\columnwidth}\raggedright\strut
The architecture is based on state-of-the-art security mechanisms and
achieves confidentiality through asymmetric cryptography. Both manifest
and payload are encrypted and then signed using COSE.\strut
\end{minipage}\tabularnewline
\begin{minipage}[t]{0.41\columnwidth}\raggedright\strut
Prevent devices from unpacking unknown formats\strut
\end{minipage} & \begin{minipage}[t]{0.53\columnwidth}\raggedright\strut
Supported through the use of the manifest format element, preconditions,
overridable/custom directives, and letting the update server decide if
there is a suitable update for a device by matching against its
registered profile.\strut
\end{minipage}\tabularnewline
\begin{minipage}[t]{0.41\columnwidth}\raggedright\strut
Specify version numbers of target firmware\strut
\end{minipage} & \begin{minipage}[t]{0.53\columnwidth}\raggedright\strut
Since devices are required to register their version alongside vendor
and class ID, operators can query device statutes from the update server
and then prepare updates matching these categories of devices.\strut
\end{minipage}\tabularnewline
\begin{minipage}[t]{0.41\columnwidth}\raggedright\strut
Enable devices to choose between images\strut
\end{minipage} & \begin{minipage}[t]{0.53\columnwidth}\raggedright\strut
Several images can be specified in a single manifest through the use of
either precursors, dependencies, URIs (mirror list), or aliases
depending on the specific use case. New options can also be defined
capturing this behaviour, such as a new option providing URL/Digest
pairs intended for parallel storage.\strut
\end{minipage}\tabularnewline
\begin{minipage}[t]{0.41\columnwidth}\raggedright\strut
Secure boot using manifests\strut
\end{minipage} & \begin{minipage}[t]{0.53\columnwidth}\raggedright\strut
The thesis has not examined bootloaders, thus this requirement is out of
scope. Investigating secure bootloaders could be part of future work in
the topic.\strut
\end{minipage}\tabularnewline
\begin{minipage}[t]{0.41\columnwidth}\raggedright\strut
Decompress on load\strut
\end{minipage} & \begin{minipage}[t]{0.53\columnwidth}\raggedright\strut
This behaviour can be encoded in the payload format of the manifest, or
through the use of a custom directive indicating a compressed payload is
stored.\strut
\end{minipage}\tabularnewline
\begin{minipage}[t]{0.41\columnwidth}\raggedright\strut
Payload in manifest\strut
\end{minipage} & \begin{minipage}[t]{0.53\columnwidth}\raggedright\strut
Can be added as the optional ``payload'' element, the value would be the
data of the payload and the regular payloadInfo element would contain
its digest. Worth noting is that for most deployments including the
payload in the manifest will dramatically increase the size of manifest,
thus a manifest parser must be prepared for it.\strut
\end{minipage}\tabularnewline
\begin{minipage}[t]{0.41\columnwidth}\raggedright\strut
Simple parsing\strut
\end{minipage} & \begin{minipage}[t]{0.53\columnwidth}\raggedright\strut
As explained in the previous section, the manifest format is simple and
easy to parse even for constrained devices.\strut
\end{minipage}\tabularnewline
\bottomrule
\end{longtable}
